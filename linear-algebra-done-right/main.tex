\documentclass{article}

% Language setting
% Replace `english' with e.g. `spanish' to change the document language
\usepackage[english]{babel}

% Set page size and margins
% Replace `letterpaper' with `a4paper' for UK/EU standard size
\usepackage[letterpaper,top=2cm,bottom=2cm,left=3cm,right=3cm,marginparwidth=1.75cm]{geometry}

% Useful packages
\usepackage{amsmath}
\usepackage{graphicx}
\usepackage{enumitem}
\usepackage[colorlinks=true, allcolors=blue]{hyperref}

\newcommand{\VEC}[2]{#1_{1},\cdots,#1_{#2}}

\title{Linear Algebra Done Right 3rd Edition Homework}
\author{Liszt Li}

\begin{document}
\maketitle

\begin{abstract}
The homework of Linear Algebra Done Right 3rd Edition.
\end{abstract}

\section*{Introduction}
No introduction
\section{chapter 1}
\subsection*{1.A}
\subsubsection*{1.A.1}

\[\frac{1}{(a+bi)} = c + di \]

\begin{equation*}
\begin{split}
    \frac{1\cdot (a-bi)}{(a+bi)(a-bi)} & = \frac{a-bi}{a^2+b^2} \\
& = \frac{a}{a^2+b^2}+ \frac{-b}{a^2+b^2}i \\
\end{split}
\end{equation*}

\[c=\frac{a}{a^2+b^2} ,d=\frac{-b}{a^2+b^2}\]

\subsubsection*{1.A.2}
\begin{equation*}
\begin{split}
    \left( \frac{-1+\sqrt{3}i}{2}\right ) ^3 &= \frac{(-1+\sqrt{3}i)^3}{8} \\
    &= \frac{(-1)^3+3\cdot (-1)^2 \cdot \sqrt{3}i + 3 \cdot -1 \cdot (\sqrt{3}i)^2 + (\sqrt{3}i)^3}{8} \\
    & = \frac{-1 + 3\sqrt{3}i + 9 + (-3\sqrt{3}i)}{8} = \frac{8}{8} \\
    & = 1
\end{split}
\end{equation*}

\subsubsection*{1.A.3}

\begin{equation*}
    \begin{split}
        i = (a+bi)^2 & = (a^2 + 2abi - b^2) \\
        & = (a^2 - b^2) + 2ab\cdot i \\
        & = 0 + i
    \end{split}
\end{equation*}
so we have:
\[a=\pm b, a\cdot b = \frac{1}{2} \]
\[a= \pm \sqrt{\frac{1}{2}}\]
then:
\[\sqrt{i} = \sqrt{\frac{1}{2}} + \sqrt{\frac{1}{2}}i\]
or
\[\sqrt{i} = -\sqrt{\frac{1}{2}} - \sqrt{\frac{1}{2}}i\]

\subsubsection*{1.A.4}
\begin{equation*}
    \begin{split}
C & ={a+bi:a,b \in R}. \\
\alpha + \beta & = a_\alpha +b_\beta i + b_\alpha + b_\beta i \\
& = b_\alpha + b_\beta i + a_\alpha +b_\beta i \\
& = \beta + \alpha
    \end{split}
\end{equation*}

\subsubsection*{1.A.10}
\begin{equation*}
    \begin{split}
        (4, -3, 1, 7) + 2x &= (5, 9, -6, 8) \\
        \implies  
        2x & = (5, 9, -6, 8) - (4, -3, 1, 7) \\
        & = (1, 12, -7, 1) \\
        \implies
        x &= (\frac{1}{2}, 6, -\frac{7}{2}, \frac{1}{2})
    \end{split}
\end{equation*}

\subsubsection*{1.A.11}
\begin{equation*}
\frac{12-5i}{2-3i} = 3+2i 
\implies
(-6+7i) \cdot (2-3i) = (-32+9i) \neq (-32-9i)
\end{equation*}

\subsection*{1.B}

\subsubsection*{1.B.1}
\begin{equation*}
    (-v) + -(-v) = 0 = -v + v \implies -(-v) = v
\end{equation*}

\subsubsection*{1.B.2}
\begin{equation*}
    a\cdot v = (av_0, av_1, \cdots av_n) = (0, 0, \cdots 0) \\
    \implies a = 0\, or\, v = (0, 0, \cdots 0)
\end{equation*}

\subsubsection*{1.B.3}
\begin{equation*}
    x = (\frac{w_1 - v_1}{3}, \frac{w_2-v_2}{3}, \cdots \frac{w_n-v_n}{3})
\end{equation*}

\subsubsection*{1.B.4}
additive inverse

\subsubsection*{1.B.5}
we have
\begin{equation*}
    0v = 0, 1v = v \implies    (0-1)v = 0 - v \implies -v + v = 0
\end{equation*}

\subsubsection*{1.B.6}
In 1.19, associativity we have $(u+v)+w = u+(v+w)$ if it's a vector space.

So If $R \cup \infty \cup -\infty$ is a vector space, we can calculate
    \[ (1 + \infty) + -\infty = \infty + -\infty = 0 \]
    \[1 + (\infty + -\infty) = 1 + 0 = 1\]
which breaks the associativity rule, so it's not a vector space.


\subsection*{1.C}

\subsubsection*{1.C.1}
\begin{enumerate}[label=(\alph*)]
\item Yes
\item No, $(0, 0, 0)$ not in subset
\item Yes
\item Yes
\end{enumerate}

\subsubsection*{1.C.3}
In 1.34, three are 3 conditions:

additive identity:

$ f \equiv 1$

close under addition

$(f+g)^{'} (-1) = f^{'} (-1) + g^{'} (-1) = 3f(2) + 3g(2) = 3(f+g)(2) $

close under scalar multiplication

$(u\cdot f)^{'}(-1) = u\cdot f^{'}(-1) = u\cdot 3f(2) = 3(u\cdot f)(2)$

\subsubsection*{1.C.4}

ONLY IF it is a subspace, then we have

\[ \int_{0}^{1}(uf) = u\int_{0}^{1}(f) = u\cdot b = b \]

so b must be 0 since u is not zero.


IF

additive identity:

$ f \equiv 1$

close under addition

$\int_{0}^{1}(f+g) = \int_{0}^{1}(f) +  \int_{0}^{1}(f) = 0 + 0 = 0 $

close under scalar multiplication

See the "only if" part

\subsubsection*{1.C.5}

No

\[ u = (1, 1) \in R^{2}, a= i \in C \]


\[ a\cdot u = (i, i) \notin R^{2} \]

\subsubsection*{1.C.6}

\begin{enumerate}[label=(\alph*)]
\item Yes

$a\in R, a^{3}=b^{3} \implies a = b $ so $(a, a, c)$ is a subspace of $R^{3}$.
\item No

we can use polar coordinate to explain why. When we calculate $\times$ in polar coordinate, it equals to extend the length, and rotate the degree anti-clock wise. So all these coordinate has the same square results:
\[(\rho, \theta + 120\cdot n) \,n \in R\]

so the following two vectors will break the "close under addition" rule:
\[((1, 90), (1, 90), (1, 0))\]
\[((1, 90), (1, 210), (1, 0))\]

because the length after sum will not be equal:  $(1, 90) + (1, 90)$ has length 2 but $(1, 90) + (1, 210)$ is shorter than 2(they are not parallel).
\end{enumerate}

\subsubsection*{1.C.7}

Make $U=\{(x, y): x, y\in N\}$, then $(0.5\cdot u) \notin U $.

\subsubsection*{1.C.8}

Make $U=\{(x, y): x\cdot y \geq 0\}$, then $(-5,0) \in U, (0,1) \in U$, but $(-5, 1) \notin U$.

\subsubsection*{1.C.9}

Clearly periodic functions meets additive identity and close under scalar multiplication. 

For addition we have 
\[(f+g)(x) = (f+g)(x+lcm(p))\]
so it's close to addition.

\subsubsection*{1.C.10}
$0 \in U_{1} \cap U_{2}$

if $ u, v \in U_{1} \cap U_{2}$, then $u+v in U_{1}$ and $U_{2}$, so $u+v \in U_{1} \cap U_{2}$

if $ u \in U_{1} \cap U_{2}$, then $a\cdot u in U_{1}$ and $U_{2}$, so $a\cdot u \in U_{1} \cap U_{2}$

\subsubsection*{1.C.11}

Just reduce 1.C.10 to all the subsets of V, and the conclusion still works.

\subsubsection*{1.C.12}

ONLY IF

Assume we have $U_{1}, U_{2}$ are two subsets of V, and make
\[U_{1} - U_{2} = A, U_{2} - U_{1} = C\]
if Both $A$ and $C$ are not blank, then we pick any $u_{a}, u_{c}$ and make $u_{a} + u_{c} = u_{ac}$ which is also in $U_{1}$ (it also works if $u_{ac}$ is in $U_{2}$, just switch $U_{1}, U_{2}$).

Since $u_{ac} and u_{a}$ are both in $U_{1}$, we have
\[u_{ac} + -u_{a} \in U_{1}\]
\[u_{a} + u_{c} + -u_{a} \in U_{1}\]
\[u_{c} \in U_{1}\]
Which is conflict with our assumuption.

\subsubsection*{1.C.15}

For any $u_{1}, u_{2} \in U$, we have $u_{1} + u_{2} \in U$, so $U+U=U$.

\subsubsection*{1.C.16}
\[U + W = \{u + w:u \in U, w \in W\} = \{w + u: w\in W, u\in U\} = W + U\]

\subsubsection*{1.C.17}
\[(U_{1} + U_{2}) + U_{3} = \{(u_{1} + u_{2}) + u_{3}: u_{1} \in U_{1}, u_{2} \in U_{2}, u_{3} \in U_{3}\} = \]

\[= \{u_{1} + (u_{2} + u_{3}): u_{1} \in U_{1}, u_{2} \in U_{2}, u_{3} \in U_{3}\} = U_{1} + (U_{2} + U_{3})\]

\subsubsection*{1.C.19}
\[U_{1} = \{(x, 0) \in R^{2}\}\]
\[U_{2} = \{(x, y) \in R^{2}\}\]
\[W = \{(0, y) \in R^{2}\}\]

\[U_{1} \neq U_{2}, U_{1}+W = U_{2}+W\].

\subsubsection*{1.C.20}

\[W = \{(0, w, 0, z) \in F^{4}: w, z\in F\}\]

\subsubsection*{1.C.21}
\[W = \{(0, 0, v, w, z) \in F^{5}: v,w,z \in F\}\]

\subsubsection*{1.C.22}
\[W_{1} = \{(0, 0, v, 0,0) \in F^{5}: v \in F\} \]
\[W_{2} = \{(0, 0, 0, w, 0) \in F^{5}: w \in F\}\]
\[W_{3} = \{(0, 0, 0, 0, z) \in F^{5}: z \in F\}\]

\subsubsection*{1.C.23}

The case in $1.C.19$ also works in this exercise.


\newpage
\section{chapter 2}
\subsection*{2.A}
\subsubsection*{2.A.1}

\begin{equation*}
    \begin{split}
w_{1} &= v_{1} - v_{2}, w_{2} = v_{2} - v_{3}, w_{3} = v_{3} - v_{4}, w_{4} = v_{4} \\
&\implies \\
v_{1} &= w_{1} + w_{2} + w{3} + w{4} \\
v_{2} &= w_{2} + w{3} + w{4} \\
v_{3} &= w{3} + w{4} \\
v_{4} &= w{4} \\
    \end{split}
\end{equation*}

\[
\forall a_{1}, a_{2}, a_{3}, a_{4} \in F \,
a_{1}v_{1} + a_{2}v_{2} + a_{3}v_{3} + a_{4}v_{4} = 
a_{1}w_{1} + (a_{1} + a_{2})w_{2}  + (a_{1} + a_{2} +a_{3})w_{3} + (a_{1} + a_{2} +a_{3}+a_{4})w_{4}
\]

\subsubsection*{2.A.3}
\[3a+2b = 5, a - 3b = 9 \implies a=3, b=-2 \implies t = 3\cdot 4 + (-2)\cdot 5 = 4 \]

\subsubsection*{2.A.5}
\begin{enumerate}[label=(\alph*)]
\item $1+i, 1-i$ point different direction in polar coordinate, and times $a\in R$ does not change the direction, so they are independent over $R$
\item We can build a counterexample

\[(1-i)(1+i) = (1+i)(1-i) = -(-1-i)(1-i) \implies (1-i)(1+i) + (-1-i)(1-i) = 0\]
\end{enumerate}

\subsubsection*{2.A.6}

if $v_{1}-v_{2}, v_{2} - v_{3}, v_{3}-v_{4}, v_{4}$ is linearly dependent, then 
\[\exists\, a_{1}(v_{1}-v_{2})+ a_{2}(v_{2} - v{3})+ a_{3}(v{3}-v{4})+ a_{4}v_{4} = 0 \]

\[\implies a_{1}v_{1} + (a_{2} - a_{1})v_{2} + (a_{3}-a_{2})v_{3} + (a_{4}-a_{3})v_{4} = 0\]

which is a counter example than $v_{1}, v_{2}, v_{3}, v_{4}$ is linearly independent

\subsubsection*{2.A.7}

IF $\exists a_{1}, a_{2}, \cdots,a_{m}$ that makes
\[a_{1}(5v_{1}-4v_{2}) + a_{2}v_{2}+\cdots + a_{m}v_{m} = 0\]
Then we have
\[5a_{1}v_{1} + (a_{2}-4a_{1})v_{2}+\cdots + a_{m}v_{m} = 0\]

which is a counter example.

\subsubsection*{2.A.8}

IF $\lambda v_{1}, \lambda v_{2}, \cdots, \lambda v_{m}$ is linearly dependent, then 
\[a_{1}\lambda v_{1} + a_{2}\lambda v_{2} + \cdots + a_{m}\lambda v_{m} = 0\]
\[(a_{1}\lambda v_{1}) + (a_{2}\lambda) v_{2} + \cdots + (a_{m}\lambda v_{m}) = 0\]
which is a counterexample that $v_{1}, v_{2}, \cdots, v_{m}$ is linearly independent.

\subsubsection*{2.A.9}

Counterexample:

\[w_{1} = -v_{1}, w_{2} = -v_{2},\cdots, w_{m} = -v_{m}\]

\subsubsection*{2.A.10}

\[a_{1}(v_{1}+w) + a_{2}(v_{2} + w)+\cdots + a_{m}(v_{m} + w) = 0\]
\[a_{1}v_{1} + a_{2}v_{2} + \cdots +a_{m}v_{m} + (a_{1}+a_{2}+\cdots +a_{m})w = 0\]
\[w=-\frac{a_{1}}{a_{1}+a_{2}+\cdots +a_{m}}v_{1} - \frac{a_{2}}{a_{1}+a_{2}+\cdots +a_{m}}v_{2} - \cdots - \frac{a_{m}}{a_{1}+a_{2}+\cdots +a_{m}}v_{m} \]

so $w\in span(v_{1},\cdots ,v_{m}$

\subsubsection*{2.A.11}

Use the process in 2.A.10 to get counterexample.

\subsubsection*{2.A.12}

$(1, 0, 0, 0, 0), (0, z, 0,0,0), (0,0,z^{2},0,0,),(0,0,0,z^{3},0),(0,0,0,0,z^{4})$ spans $P_{4}(F)$, so the length of spanning list is 5; if six polynomials is linearly independent, it's conflict with 2.23: "Length of linearly independent list $\leq$ length of spanning list"

\subsubsection*{2.A.13}

also we can leverage 2.23. It's easy to see that
\[(1, 0, 0, 0, 0), (0, z, 0,0,0), (0,0,z^{2},0,0,),(0,0,0,z^{3},0),(0,0,0,0,z^{4})\]
is linearly independent, and it has length 5; if 4 polynomials spans $P_{4}(F)$, it's conflict with 2.23

\subsubsection*{2.A.14}
\begin{enumerate}[label=(\alph*)]
\item IF:

$\forall m$, $\exists $ linearly independent vectors in $V$, so there is no number $m$ than make $m$ bigger than the length of all linearly independent vectors. 
\item ONLY IF :

For any $m$, we pick m vectors $v_{1}, v_{2}, \cdots, v_{m}$ from $V$, then the $span(v_{1}, v_{2},\cdots,v_{m})\in V$. Then we pickup any $v_{m+1}\in V-span(v_{1}, v_{2},\cdots,v_{m})$, we have vectors $v_{1}, v_{2}, \cdots, v_{m+1}$ are linear independent, based on 2.21.a: if it's linearly dependent, the $v_{m+1}\in span(v_{1}, v_{2},\cdots,v_{m})$.
\end{enumerate}

\subsubsection*{2.A.17}
Copy the answer from \href{https://linearalgebras.com/2a.html}{here}.

Let's think $P_{m}(F)$, the length of its span vectors in $m+1$, so $(z, p_{0}(z), p_{1}(z), \cdots, p_{m}(z))$ is linearly dependent because this list has length $m+2$. 

If $(p_{0}(z), p_{1}(z), \cdots, p_{m}(z))$ is linearly independent, then we have $z\in span(p_{0}(z), p_{1}(z), \cdots, p_{m}(z))$(by problem 11), so we have

\[z+a_{0}p_{0}(z) + a_{1}p_{1}(z) +\cdots + a_{m}p_{m}(z) = 0\]

Since $p_{j}(2) = 0$, we got 2 = 0 in the previously equation, which is not possible.

\subsection*{2.B}
\subsubsection*{2.B.1}

${0}$

\subsubsection*{2.B.3}

\begin{enumerate}[label=(\alph*)]
\item a basis of $U$ is

\[(1, \frac{1}{3}, 0, 0, 0), (0, 0, 1, \frac{1}{7}, 0), (0, 0, 0, 0, 1)\]

\item extend to 

\[(1, \frac{1}{3}, 0, 0, 0), (0, 0, 1, \frac{1}{7}, 0), (0, 0, 0, 0, 1), (0, 1, 0, 0, 0), (0, 0, 0, 1, 0)\]

\item 

\[W = {(0, x, 0, y, 0) \in R^{5} x, y \in R}\]

\end{enumerate}

\subsubsection*{2.B.4}

\begin{enumerate}[label=(\alph*)]
\item a basis of $U$ is

\[(1, 6, 0, 0, 0), (0, 0, 1, \frac{1}{2}, 0), (0, 0, 1, 0, \frac{1}{3})\]

\item extend to 

\[(1, 6, 0, 0, 0), (0, 0, 1, \frac{1}{2}, 0), (0, 0, 1, 0, \frac{1}{3}), (0, 1, 0, 0, 0), (0, 0, 1, 0, 0)\]

\item 

\[W = {(0, x, y, 0, 0) \in R^{5} x, y \in R}\]

\end{enumerate}

\subsubsection*{2.B.5}

The basis does not exists because we have the length of independent list is $\leq$ length of spanning list, here we have a spanning list

\[(1, 0, 0, 0), (0, 0, 1, 0), (0,0,0, 1)\] 

so the list with length 4 can't be a independent list.

\subsubsection*{2.B.6}

\[v_{1}+v_{2}, v_{2}+v_{3}, v_{3}+v_{4}, v_{4}\]

Independent, see $2.A.6$

Spanning list

\[ \forall v\in V, v =  a_{1}v_{1} + a_{2}v_{2} + a_{3}v_{3} + a_{4}v_{4}\]
we also have 
\[ v = a_{1}(v_{1} + v_{2}) + (a_{2} - a_{1})(v_{2} + v_{3}) + (a_{3} -a_{2}+a_{1})(v_{3}+v_{4}) + (a_{4} - a_{3} + a_{2}-a_{1})v_{4} \]

so the list spans $V$.

\subsubsection*{2.B.7}

counterexample

consider this vector list:

\[(1,0,0,0), (0,1,0,0),(0,0,1,0),(0,0,0,1)\]

apparently, the list spans $F^{4}$. Consider $U=span((1,0,0,0), (0,1,0,0), (0,0,1,1)$, we have

$(0,0,1,1) \in U$ but not in $span(v_{1}, v_{2})$.


\subsubsection*{2.B.7}

$u_{1},\cdots, u_{m}, w_{1}, \cdots,w_{n}$ is basis.

\textbf{Independent}

if not independent, then exists
\[a_{1}u_{1}+\cdots+a_{m}u_{m} + b_{1}w_{1} + \cdots +b_{n}w_{n} = 0\]
\[a_{1}u_{1}+\cdots+a_{m}u_{m} = -b_{1}w_{1} - \cdots -b_{n}w_{n}\]

which is conflict the define of direct sum.

\textbf{Spanning list}

$\forall v\in U+W$, $v = u+w\, u\in U, w\in W$

so $\exists a_{1},\cdots,a_{m}, b_{1},\cdots, b_{n}$ that makes

\[ v = a_{1}u_{1}+\cdots+a_{m}u_{m} + b_{1}w_{1} + \cdots +b_{n}w_{n}\]

so $u_{1},\cdots, u_{m}, w_{1}, \cdots,w_{n}$ spans $V$.

\subsection*{2.C}
\subsubsection*{2.C.1}

Let's assume $dim V = n$, so the basis of $U$ is 
\[u_{1}, u_{2}, \cdots, u_{n}\]

If $U\neq V$, then there must be a vector $v\in V-U$, now consider the list

\[u_{1}, u_{2}, \cdots, u_{n}, v\]

it's easy to know that it is independent, but its length is $n+1$ which is larger than $dim V$, so that breaks the law that length of independent $\leq$ length of spanning list which is $n$.

\subsubsection*{2.C.2}

since $R^{2}$ has basis with length 2, so the dim of subspace of $R^{2}$ can be 0, 1 or 2, which is $\{0\}, R^{2}$ or the lines in $R^{2}$. Since $\{0\}$ must be in any subspaces, so all lines must through origin.

\subsubsection*{2.C.4}
\begin{enumerate}[label=(\alph*)]
\item $(x-6), x(x-6), x^{2}(x-6), x^{3}(x-6)$

\item $1, (x-6), x(x-6), x^{2}(x-6), x^{3}(x-6)$
\item $W = span(1)$, so $W=\{c:c \in F\}$
\end{enumerate}

\subsubsection*{2.C.5}
\begin{enumerate}[label=(\alph*)]
\item $1, x, (x-6)^{3}, (x-6)^{4}$
\item $1, x, x^{2}, (x-6)^{3}, (x-6)^{4}$
\item $W = span(x^{2})$, so $W=\{cx^{2}:c \in F\}$
\end{enumerate}

\subsubsection*{2.C.6}
\begin{enumerate}[label=(\alph*)]
\item $1, (x-2)(x-5), x(x-2)(x-5), x^{2}(x-2)(x-5)$
\item $1, x, (x-2)(x-5), x(x-2)(x-5), x^{2}(x-2)(x-5)$
\item $W = span(x)$, so $W=\{cx:c \in F\}$
\end{enumerate}

\subsubsection*{2.C.7}
\begin{enumerate}[label=(\alph*)]
\item $1, (x-2)(x-5)(x-6), x(x-2)(x-5)(x-6)$
\item $1, x, x^{2}, (x-2)(x-5)(x-6), x(x-2)(x-5)(x-6)$
\item $W = span(x, x^{2})$, so $W=\{cx+dx^{2}:c \in F, d \in F\}$
\end{enumerate}

\subsubsection*{2.C.8}
\begin{enumerate}[label=(\alph*)]
\item if $\int^{1}_{-1}f(x)=0$, means $f(x) = -f(-x)$, so the basis of $U$ is $1, x, x^{3}$
\item $1, x, x^{2}, x^{3}, x^{4}$
\item $W = span(x^{4}, x^{2})$, so $W=\{cx^{4}+dx^{2}:c \in F, d \in F\}$
\end{enumerate}

\subsubsection*{2.C.9}

We only have to find a independent list with length $m-1$ in $span(v_{1}+w, \cdots,v_{m}+w)$. The following list is an example:
\[v_{m}-v_{1}, v_{m}-v_{2}, \cdots, v_{m}-v_{m-1}\]

It is independent because $v_{1}, \cdots, v_{m}$ is independent.

\subsubsection*{2.C.10}

We need to prove that the list spans $P(F)$ and it's independent. 

For any $p\in P(F)$, we can start from the scalar of the highest degree $z^{m}$, use $p_{m}$ to guarantee we get the correct $a_{m}$; Then we got the correct $a_{m-1}$ using $z_{m-1}$ in $p_{m}$ and $p_{m-1}$; keep doing this until we got $p_{0}$, so we have the list spans $P(F)$.

In this process, if the target vector is $0$, we can get that all the scalars are $0$, so the list is independent.

\subsubsection*{2.C.11}

we have 
\[dim(U+W) = dim(U) + dim(W) - dim(U\cap W)\]
so we have $dim(U\cap W)=0$, which means the two subspaces has no common vectors except $0$, then we have $U\oplus W$.

\subsubsection*{2.C.12}
\[dim(U\cap W) = 5 + 5 - 9 = 1\]
so $U\cap W \neq \{0\}$

\subsubsection*{2.C.13}

\[dim(U\cap W) = 4 + 4 -6 = 2\]
so if $v_{1}, v_{2}$ is a basis of $U\cap W$, then they are independent, so neither of $v_{1}, v_{2}$ is a scalar multiple of the other.

\subsubsection*{2.C.14}

Make $U_{1\_m} = U_{1} + \cdots + U_{m}$, then we have 
\[dim(U_{1\_m}) \leq dim(U_{1\_m-1})+dim(U_{m}) \leq dim(U_{1\_m-2}) + dim(U_{m-1}) + dim(U_{m}) \leq \cdots \leq dim U_{1} + \cdots + dim U_{m}\]

\subsubsection*{2.C.15}

If $v_{1}, \cdots , v_{n}$ is a basis of $V$, then $\forall 1\leq i, j \leq n$, we have $v_{i}, v_{j}$ are independent. So $span(v_{i}) \cap span(v_{j})=\{0\}$ and $v_{i}, v_{j}$ is the basis of the subspace.

If we add $U_{i}=span(v_{i}$ together, we got a $n$ dimension vector space, with $v_{1}, \cdots, v_{n}$ as its basis, which is $V$.

\subsubsection*{2.C.16}

Make $U_{1\_m} = U_{1} \oplus \cdots \oplus U_{m}$, then we have 
\[dim(U_{1\_m}) = dim(U_{1\_m-1})+ dim(U_{m}) = dim(U_{1\_m-2}) + dim(U_{m-1}) + dim(U_{m}) = \cdots = dim U_{1} + \cdots + dim U_{m}\]
\newpage
\section{chapter 3}
\subsection*{3.A}
\subsubsection*{3.A.1}

If $T$ is linear, make $u=(x, y, z)$ and $v=2u$, then we should have $Tv = 2Tu$
\[(4x-8y+6z+b, 12x+c\cdot 8xyz) = 2\cdot (2x-4y+3z+b, 6x+cxyz)\]

then we have $b=c=0$

If $b=c=0$

\begin{equation*}
    \begin{split}
T(u+v) &= (2(x_{u}+x_{v}) - 4(y_{u}+y_{v}) + 3(z_{u}+z_{v}), 6(x_{u}+x_{v}) \\
&= (2x_{u}-4y_{u}+3z_{u}, 6x_{u}) + (2x_{v}-4y_{v}+3z_{v}, 6x_{v}) \\
&= Tu+Tv \\
T(\lambda u) &=(2\lambda x - 4\lambda y + 3\lambda z, 6\lambda x) \\
&= \lambda (2x-4y+3z, 6x) \\
&= \lambda Tu
\end{split}
\end{equation*}

\subsubsection*{3.A.2}

If $T$ is linear, then for $p_{0}, p_{1}$, $T(p_{0}+p_{1}) = Tp_{0} + Tp_{1}$, we must have 
\[b(p_{0}(1)+p_{1}(1))(p_{0}(2)+p_{1}(2)) = b(p_{0}(1)p_{0}(2)+p_{1}(1)p_{1}(2))\]

Expand the left part, to make the equation work, we have $b=0$

For any $\lambda \in R$,
\[c\lambda \sin{p(0)} = c \sin{\lambda p{0}}\]

make $\lambda = 2$, to make the equation work, $c=0$.

If $b=c=0$, we can see that the $T$ is the linear combination by linear maps in $3.4$, so $T$ is also linear.

\subsubsection*{3.A.4}

If $v_{1}, \cdots , v_{m}$ is dependent, then $ \exists a_{1}, \cdots, a_{m} $ that 
\begin{equation*}
    \begin{split}
0 &= a_{1}v_{1} + \cdots + a_{m}v_{m} \\
0 &= T(0) \\
 &= T(a_{1}v_{1} + \cdots + a_{m}v_{m}) \\
    & = Ta_{1}v_{1} + \cdots + Ta_{m}v_{m} \\
    & = a_{1}Tv_{1} + \cdots + a_{m}Tv_{m}
\end{split}
\end{equation*}

Which is a counter example of $Tv_{1}, \cdots, Tv_{m}$ is independent.

\subsubsection*{3.A.7}

If $dim(V) = 1$, let's assume $v$ is a basis of $V$, based on the definition of span, then we have 
\[V=\{av\, a\in F\}\]

IF $T\in L(V,V)$, then we have $Tv\in V$, so $Tv = \lambda v$.

\subsubsection*{3.A.8}

\[\varphi(a_{0}, a_{1}) = \sqrt{a_{0}^{2} + a_{1}^{2}}\]

\subsubsection*{3.A.9}
\[\varphi(i, j) = (i, 0)\]

\subsubsection*{3.A.10}

$T$ is not linear, because it does not fulfill $T(u+v) = Tu + Tv$:

\[u\in U, v \in (V - U)\] 

based on the defination, we have 

\begin{equation*}
    \begin{split}
u+v &\in (V-U) \\
Tu &\neq 0  \\
Tv &= 0 \\
T(u+v) &= 0 \\
T(u+v) &\neq Tu + tv
\end{split}
\end{equation*}

So $T$ is not linear.

\subsubsection*{3.A.11}

$U$ is a subspace of $V$, so we can find a basis of $U$: $u_{1}, \cdots, u_{n}$, and we can expand the basis of $U$ to basis of $W$ to $u_{1}, \cdots, u_{n}, q_{1}, \cdots, q_{m} $.

Make 
\[Q = span(q_{1}, \cdots, q_{m})\]

and $R\in L(Q, W)$.

Define the $T$ like:

\[Q = span(q_{1}, \cdots, q_{m})\]

 \[
    Tv=\left\{
 \begin{array}{ll}
   Sv    &    v\in U \\
   Rv    &    v \in Q \\
   Sv_{0} + Rv_{1}    &    v = v_{0} + v_{1}, v_{0} \in U, v_{1} \in Q
  \end{array}
  \right.
  \]

\subsubsection*{3.A.12}

If $v_{1},\cdots, v_{n}$ is basis of $V$, and $S_{1}, \cdots, S_{m}$ is basis of $L(V, W)$, then we have 
\[W=span(S_{1}v_{1}, \cdots, S_{1}v_{n}, \cdots, S_{m}v_{1},\cdots, S_{m}v_{n})\]
which is conflict with that $W$ is infinite dimensional.

\subsubsection*{3.A.13}

$v_{1},\cdots,v_{m}$ is linearly dependent, so $\exists v_{i}$ that satisfy:
\[v_{i} = a_{1}v_{1} + \cdots + a_{i-1}v_{i-1} + a_{i+1}v_{i+1} + \cdots + a_{m}v_{m}\]

so if we use any $T\in L(v, W)$, the $T^{'}$ to convert $v_{1}, \cdots, v_{m}$ to following vectors does not exists for any  $w\in W,w\neq 0$:

\[Tv_{1}, \cdots, Tv_{i-1}, w + (aTv_{1}  + \cdots + a_{i-1}Tv_{i-1} + a_{i+1}Tv_{i+1} + \cdots + a_{m}Tv_{m}), Tv_{i+1}, \cdots, Tv_{m}\]


\subsubsection*{3.A.14}
 
Here is a counter example:

\begin{equation*}
    \begin{split}
T(x_{0}, \cdots, x_{n}) &= (x_{0}+x_{1}, x_{1}, \cdots, x_{n}) \\
S(x_{0}, \cdots, x_{n}) &= (x_{0}, x_{0}+ x_{1}, \cdots, x_{n})
\end{split}
\end{equation*}

\subsection*{3.B}
\subsubsection*{3.B.1}

\[T(x_{1}, x_{2}, x_{3}, x_{4}, x_{5}) = (x_{4}, x_{5})\]

\subsubsection*{3.B.2}
\[(ST)^{2}(v) = S(T(ST(v)))\]
Since range $S\subset null T$, so we have $T(ST(v)) = 0$, implies $(ST)^{2} = 0$.

\subsubsection*{3.B.3}

\begin{enumerate}[label=(\alph*)]
\item $range T = V$
\item $null T = 0$
\end{enumerate}


\subsubsection*{3.B.4}

We have an count example:

\begin{equation*}
    \begin{split}
T_{1}(x_{1}, x_{2}, x_{3}, x_{4}, x_{5}) &= (x_{1}, 0, 0, 0) \\
T_{2}(x_{1}, x_{2}, x_{3}, x_{4}, x_{5}) &= (0, x_{2}, 0, 0) \\
T_{3}(x_{1}, x_{2}, x_{3}, x_{4}, x_{5}) &= (0, 0, x_{3}, 0) \\
T_{4}(x_{1}, x_{2}, x_{3}, x_{4}, x_{5}) &= (0, 0, 0, x_{4}) \\
T_{1} + T_{2} + T_{3} + T_{4} &= (x_{1}, x_{2}, x_{3}, x_{4})
\end{split}
\end{equation*}

The last $null T = (5-5) = 1$, so it's not the subspace.

\subsubsection*{3.B.6}

$dim(T) = range T + null T, range T = null T \implies null T = 2.5$, which make no sense.

\subsubsection*{3.B.9}

If $Tv_{1}, \cdots, Tv_{n}$ is dependent, then we have $a_{1}, \cdots, a_{n}$ that:
\[a_{1}Tv_{1} + \cdots +a_{n}Tv_{n} = T(a_{1}v_{1}+\cdots+a_{n}v_{n}) = 0\]

Since T is injective, and $T(0) = 0$, implies $a_{1}v_{1}+\cdots+a_{n}v_{n} = 0$, which is a counter example of they are linearly independent.

\subsubsection*{3.B.10}

For $\forall w\in range T$, there exists a $v\in V, Tv=w$. Since $v_{1}, \cdots, v_{n}$ spans $V$, we have 
\[v = a_{1}v_{1} + \cdots + a_{n}v_{n}\]

So 

\[w = Tv = Ta_{1}v_{1} + \cdots + Ta_{n}v_{n} = a_{1}Tv_{1} + \cdots + a_{n}Tv_{n}\]

indicates that $Tv_{1}, \cdots, Tv_{n}$ spans range $T$.

\subsubsection*{3.B.11}


\begin{equation*}
    \begin{split}
S_{1}\cdots S_{n}(u) &= S_{1}\cdots S_{n}(v) \implies \\
S_{2}\cdots S_{n}(u) &= S_{2}\cdots S_{n}(v) \implies \\
&\cdots \\
S_{n}(u) &= S_{n}(v) \implies \\
u &= v
\end{split}
\end{equation*}

\subsubsection*{3.B.12}

Make $u_{1}, \cdots, u_{n}$ a basis of $null T$, then we can expand it to $u_{1}, \cdots, u_{n}, w_{1}, \cdots, w_{m}$ to the basis of $V$.

$U=span(w_{1},\cdots, w_{m}$ is the subspace we need.

\subsubsection*{3.B.13}

$dim(null T) = 2$, so $dim(range T) = 4 - 2 = 2$, so T is surjective.

\subsubsection*{3.B.15}

the target space has dimension $2$, and $dim(F^{2}) \leq 2$, so $range(null T) + dim(T) \leq 4$ which is less than 5, so the transaction does exists.

\subsubsection*{3.B.16}

Since $null T, range T$ are finite dimensional, assume $u_{1}, \cdots, u_{n}$ is a basis of $null T$ and $v_{1},\cdots, v_{m}$ is a basis of $range T$, then there is a vector $w_{1},\cdots,w_{m}$ defined by
\[Tw_{1} = v_{1}, \cdots, Tw_{m} = v_{m}\]

then we have $u_{1}, \cdots, u_{n}, w_{1}, \cdots, w_{m}$ is a basis of $V$. 

If not, there $\exists v_{0}$ that $u_{1}, \cdots, u_{n}, w_{1}, \cdots, w_{m}, v_{0}$ is linear independent, which indicate that
\[ Tu_{1}, \cdots, Tu_{n}, Tw_{1}, \cdots, Tw_{m}, Tv_{0} = 0, \cdots, 0, v_{1}, \cdots, v_{m}, Tv_{0}\] 

is independent, which is conflict that $v_{1},\cdots, v_{m}$ is a basis of $range T$.

\subsubsection*{3.B.17}

$dim\, V = n, dim\, W = m$, If $n \leq m$, then length of basis $W=w_{1},\cdots,w_{m}$ is longer than basis $V=v_{1},\cdots,v_{n}$, then we can define a injective map as:
\[Tv_{1} = w_{1}, \cdots, Tv_{n} = w_{n}\]

When there exists an injective map, we have $dim\,V\leq dim\,W$ using $3.23$.

\subsubsection*{3.B.18}

if $dim\,V\geq dim\,W$, then any basis of  $V=v_{1},\cdots,v_{n}$ is longer than the basis of $W=w_{1}, \cdots,w_{m}$ ($n\geq m)$. So here in a surjective map:

\begin{equation*}
    \begin{split}
Tv_{1} & = w_{1} \\
& \cdots  \\
Tv_{m} &= w_{m} \\
tv_{m+1} &= 0 \\
&\cdots \\
tv_{n} &= 0
\end{split}
\end{equation*}

If there is a surjective map, and
\[dim\,V = dim(null\,T) + dim(T) = dim(null\, T) + dim\,W\]
which implies $dim\,V\geq dim\,W$.

\subsubsection*{3.B.19}

If $null\,T=U$, we have $dim(range\,T) = dim\,V - dim\,U \leq dim\,W$, the $=$ works only T is surjective.
since $U$ is a subspace, we can write a basis of $V$ as:
\[u_{1}, \cdots, u_{n}, v_{1},\cdots, v_{m}\]

If $W$ has lower dimension $x\leq dim\,V-dim\,U$, then we can build a map that make $null\,T=U$
\begin{equation*}
    \begin{split}
    Tu_{1} &= 0 \\
    & \cdots \\
    Tu_{n} &= 0 \\
Tv_{1} & = w_{1} \\
& \cdots  \\
Tv_{x} &= w_{x} \\
tv_{x+1} &= w_{1} \\
&\cdots \\
tv_{m} &= w_{1}
\end{split}
\end{equation*}

\subsubsection*{3.B.20}

If $T$ is injective, then define $S$ as $S(Tv) = v:v\in V$, so $STv = S(Tv) = v = Iv \implies ST$ is the identity map.

If $T$ is not a one-one map, then $\exists\, Tv_{1} = Tv_{2} = s$, and $Ss$ can't be both $v_{1}$ and $v_{2}$, so no $ST = I$ exists.

\subsubsection*{3.B.21}

IF $ST = I$ and $T$ is not surjective, then $\exists\,w \in W, w\notin range\,T$, so $T(Sw)\notin W$, which is a counter example of $ST = I$.

If T is surjective, then $\forall\,w \in W, \exists\,v\in V, Tv = w$, so let's define $S(Tv) = v: Tv\in W$, and we have $ST = I$.

\subsubsection*{3.B.22}

If dim null (range $T$) = dim null $S$, then 
\[dim\, null\, ST = dim\, null\, S + dim\,null\,T\]

else 

\[dim\, null\, ST < dim\, null\, S + dim\,null\,T\]

\subsubsection*{3.B.23}

Because $T(U) \subseteq V$, so $range\,ST\subseteq range\,S$, so 
\[dim\,range\,ST\leq dim\,range\,S\]

$X = range\, T$ is also a vector spaces, and we $S\in L(X, w)$, we have 
\[dim X = dim\, null\,S + dim\,range S\]

so 
\[dim\,range S \leq dim\,X\]
\[dim\,range ST \leq dim\,range T\]

finally, we have
\[dim\,range\,ST\leq min\{dim\,range\,S,\, dim\,range\,T\}\]

\subsubsection*{3.B.24}

If $T_{2} = ST_{1}$, and $v_{1},\cdots,v_{n}$ is a basis of $null\, T_{1}$, then we have $\forall i \in [1,\cdots,n] T_{2}v_{i}=0$ because $T(0) = 0$, so
    null $T_{1}\subset$ null $ T_{2}$.
    
If null $T_{1} \subset$ null $T_{2}$, suppose $u_{1},\cdots,u_{n}$ is a basis of null $T_{1}$,  $w_{1}, \cdots,w_{m}$ is a basis of null $T_{2} (n\leq m)$, extend both basis of null space to the basis of $V$, we have 2 basis:
\[u_{1},\cdots,u_{n},\cdots, u_{v}\]
\[w_{1},\cdots,w_{n},\cdots,w_{m}, \cdots, w_{v}\]

It's easy to prove(2.39) that $u_{n+1},\cdots,u_{v}$ is the basis of range $T_{1}$ and $w_{m+1},\cdots,w_{v}$ is the basis of range $T_{2}$, so we build a mapping $S$ that

\begin{equation*}
    \begin{split}
    ST_{1}u_{n+1} &= 0 \\
    &\cdots \\
    ST_{1}u_{m} &= 0 \\
    ST_{1}u_{m+1} &= T_{2}w_{m+1} \\
    &\cdots \\
    ST_{1}u_{v} &= T_{2}w_{v}
\end{split}
\end{equation*}

\subsubsection*{3.B.25}

If $\exists S\in L(V, V)$ and $T_{1} = T_{2}S$, then we have range $S \subset V$, so $T_{2}(S) \subset T_{2}(V)$, aka range $T_{1} \subset$ range $T_{2}$.

If range $T_{1}\subset $range $T_{2} \subset W$, and $w_{1}, \cdots, w_{n}$ is a basis of range $T_{1}$. 
This basis can be extend to the basis of range $T_{2}$ like $w_{1}, \cdots, w_{n}, \cdots, w_{m}$, and finally to the basis of $W$: $w_{1},\cdots,w_{m},\cdots, w_{w}$.

For $T_{1}$, we have:
\begin{equation*}
    \begin{split}
T_{1}u_{1} &= w_{1} \\
&\cdots \\
T_{1}u_{n} &= w_{n} \\
T_{1}u_{n+1} &= 0 \\
&\cdots \\
T_{1}u_{v} &= 0 
\end{split}
\end{equation*}

In the equation, $v$ is the dimension of $V$. and for $T_{2}$:
\begin{equation*}
    \begin{split}
T_{2}v_{1} &= w_{1} \\
&\cdots \\
T_{2}v_{m} &= w_{m} \\
T_{2}v_{m+1} &= 0 \\
&\cdots \\
T_{2}v_{v} &= 0 \\
\end{split}
\end{equation*}

so the target map $S$ would be:
\begin{equation*}
    \begin{split}
Su_{1} &= v_{1} \\
&\cdots \\
Su_{n} &= v_{n} \\
Su_{n+1} &= 0 \\
&\cdots \\
Su_{v} &= 0 
\end{split}
\end{equation*}

because 
\[T_{2}S(u_{1}) = T_{2}(v_{1}) = w_{1} = T_{1}u_{1} \] for all $u$.

\subsubsection*{3.B.27}

For a polynomial p with deg n, we write its vector as 
\[a_{1}, \cdots, a_{n}\]

and for polynomial q with deg $n+1$, we write its vector as
\[b_{1}, \cdots, b_{n+1} \]

then $q^{'}$ has deg n:
\[b_{2}, 2b_{3}, 3b_{4},\cdots, nb_{n+1}\]

and $q^{''}$ has deg $n-1$:

\[2b_{3}, 6b_{4}, 12b_{5},\cdots, n*(n-1)b_{n+1}\]

so $q$ can be defined as:

\begin{equation*}
    \begin{split}
    a_{n} &= 3nb_{n+1} \\
    a_{n-1} &= 3(n-1)b_{n} + 5n*(n-1)b_{n+1} \\
    &\cdots \\
    a_{1} & = 3b_{2} + 52b_{3}
\end{split}
\end{equation*}

we can calculate $b_{n+1},\cdots, b_{1}$ top-down.

\subsubsection*{3.B.28}

$w_{1}, \cdots, w_{m}$ is a basis of range $T$, so we have $v_{1},\cdots, v_{m}$ that $\{Tv_{i} = w_{i}, i\in [1, 2,\cdots, m]$. 
It's easy to prove that $v_{1}, \cdots, v_{m}$ is independent(if not, $w_{1}, \cdots, w_{m}$ will also be dependent). Now expand the vector to a basis of $V: v_{1},\cdots, v_{m},\cdots, v_{n}$.

$\forall v, \exists a_{1},\cdots, a_{m}\in F$ that:
\[v = a_{1}v_{1} + \cdots + a_{m}v_{m}\]

also $a_{1},\cdots,a_{m}$ is uniq for any $v\in V$.

Now define

\begin{equation*}
    \begin{split}
\varphi_{1} &= a_{1} \\
\varphi_{2} &= a_{2} \\
&\cdots \\
\varphi_{m} &= a_{m}
\end{split}
\end{equation*}

$a_{1}, \cdots, a_{m}$ is the scala in the equation.

\subsubsection*{3.B.29}

dim $F = 1$, so a basis of $V$ is like
\[v_{1},\cdots,v_{n}, v_{n+1}\]

and $v_{1},\cdots, v_{n}$ is the basis of $\varphi$. 

Now consider the span of $v_{n+1} = U$, it's easy see $U \cap $ null $\varphi = \{0\}$ and $U + $ null $\varphi = V$. because

\[U = \{av_{n+1}: a\in F\} = \{au: a\in F\]
so $V = $null $\varphi \oplus \{au:a \in F\}$.

\subsubsection*{3.B.30}
Use the proof of 3.B.29, if $v_{n+1}\in $basis of $V,\varphi_{1}v_{n+1} = a_{1} \neq 0, \varphi_{2}v_{n+1} = a_{2} \neq 0$,  So 
\begin{equation*}
    \begin{split}
\varphi_{1}v_{n+1} &= a_{1} \\
\varphi_{2}v_{n+1} &= a_{2} \\
\frac{\varphi_{1}}{\varphi_{2}} & = \frac{a_{1}}{a_{2}} \\
\varphi_{1} &= c\varphi_{2}
\end{split}
\end{equation*}

\subsubsection*{3.B.31}
\[T_{1}(a_{1}, \cdots, a_{5}) = (a_{4}, a_{5})\]
\[T_{2}(a_{1}, \cdots, a_{5}) = (a_{5}, a_{4})\]

\subsection*{3.C}
\subsubsection*{3.C.1}

If the less than range $T$ nonzero entries, then in the matrix defined in 3.30, at least one column $k$ will be all zero, then in 3.32, the $Tv_{k}$ would be zero, so there would be less than dim range $T$ nonzero vectors in $TV$: they can't span a space with dim range $T$.

\subsubsection*{3.C.2}

basis of $P_{3} = (0, 0, 0, 1), (0, 0, 1, 0), (0, 1, 0, 0), (1, 0, 0,0)$ 

basis of $P_{2} = (3, 0, 0), (0, 2,  0), (0, 0, 1)$

\subsubsection*{3.C.3}

Suppose $u_{1}, \cdots, u_{n}$ is a basis of null $T$, then we can extend the vector to a basis of $V$: $v_{1},\cdots, v_{m}, u_{1}, \cdots, u_{n} $, also we have dim range $T = m$, and $Tv_{1}, \cdots, Tv_{m}$ is a basis of range $T$.

so here is the $m \times (m+n)$ matrix  we need:

\[
\begin{pmatrix}
1 & 0 & \cdots & 0 & \cdots & 0\\
0 & 1 & \cdots & 0 & \cdots & 0\\
\cdots \\
\cdots &  & & 1 & \cdots & 0 
\end{pmatrix}
\]

\subsubsection*{3.C.4}

If $Tv_{1} = 0$ , then entries of the 1st column are all 0; If no, $w_{1} = Tv_{1}$, and extend $w_{1}$ to a basis of $W$, then the entries of 1st column have a initial 1 followed by 0.

\subsubsection*{3.C.5}

If $w_{1} \notin $ range $T$, then every entity in the 1st row of $M$ has to be 0; 

If $w_{1} \in $ range $T$, then $\exists v_{1} \in V, Tv_{1} = w_{1}$, then any basis with $v_{1}$ as the 1st element make the 1st row has a initial 1 followed by 0.

\subsubsection*{3.C.6}

Suppose $v_{1}, \cdots, v_{n}$ is a basis of $V$ and $w_{1} = Tv_{1} \neq 0$, it's easy to know that $v_{1}, v_{2} + v_{1}, \cdots, v_{n}+v_{1}$ is also a basis. Because dim range $T = 1$, so we have

\begin{equation*}
    \begin{split}
    Tv_{1} & = w_{1} \\
    T(v_{2} + v_{1}) & = a_{2}w_{1} \\
    & \cdots \\
    T(v_{n} + v_{1}) & = a_{n}w_{1}
\end{split}
\end{equation*}

So the basis of $V$ is $v_{1},  \frac{v_{2} + v_{1}}{a_{2}}, \cdots, \frac{v_{n}+v_{1}}{a_{n}}$ and the basis of $W$ is $w_{1}$.


If we have a $M$ with all entries equals to 1, then there is a basis of $V$ and $W$ that $Tv = w, v\in$ basis $V$, then it's easy to know every $v\in V$ is a linear combination of basis $V$, so $Tv$ is linear combination of $w$, so $w$ is a basis of range $T$, which means than dim range $T = 1$.

\subsubsection*{3.C.10}

In $3.47$ we learn that
\[(AC)_{j,k} = A_{j,.}C_{.,k}\]

so 

\begin{equation*}
    \begin{split}
    (AC)_{j,.} &= [(AC)_{j,1}, \cdots, (AC)_{j,k}] \\
    &= [A_{j,.}C_{.,1}, \cdots, A_{j,.}C_{,.k}] \\
    &= A_{j,.}[C_{.,1}, \cdots, C_{.,k}] \\
    &= A_{j,.}C
\end{split}
\end{equation*}

\subsubsection*{3.C.11}

\begin{equation*}
    \begin{split}
aC &= [\sum_{r=1}^{n}a_{r}C_{r,1}, \sum_{r=1}^{n}a_{r}C_{r,2},\cdots, \sum_{r=1}^{n}a_{r}C_{r,p}] \\
&= \sum_{r=1}^{n}[a_{r}C_{r,1}, a_{r}C_{r,2}, \cdots, a_{r}C_{r,4}] \\
&= \sum_{r=1}^{n}a_{r}C_{r,.} \\
&= a_{1}C_{1,.} +\cdots + a_{n}C_{n,.}
\end{split}
\end{equation*}

\subsubsection*{3.C.14}

In $3.47,3.49, 3.C.10$ we learn that

\begin{equation*}
    \begin{split}
(AC)_{j,k} &= A_{j,.}C_{.,k} \\
(AC)_{.,k} &= AC_{.,k} \\
(AC)_{k,.} &= A_{k,.}C
\end{split}
\end{equation*}
so 

\begin{equation*}
    \begin{split}
    A(BC)_{j,k} &= A_{j,.}(BC)_{.,k} \\
    & = A_{j,.}(BC_{.,k}) \\
    (AB)C_{j,k} &= (AB)_{j,.}C_{.,k} \\
     &= (A_{j,.}B)C_{.,k}
\end{split}
\end{equation*}

expand both $A(BC)_{j,k}$ and $(AB)C_{j,k}$, they all equals to (if $B$ is $m*n$ matrix):
\[\sum_{r=1}^{m}\sum_{s=1}^{n}A_{j,m}B_{m,n}C_{n,k}\]

so $A(BC) = (AB)C$

\subsection*{3.D}
\subsubsection*{3.D.1}

Because $S$ and $T$ are both invertible, then we have
\[ST\times T^{-1}S^{-1} = S(TT^{-1})S^{-1} = SIS^{-1} = I\]

and 
\[T^{-1}S^{-1} \times ST = T^{-1}(S^{-1}S)T = T^{-1}IT = I\]

so $ST$ is invertible and $(ST)^{-1} = T^{-1}S^{-1}$

\subsubsection*{3.D.2}

Suppose $v_{1}, \cdots, v_{n}$ is a basis of $V$, consider operators:
\[T_{1}v_{1} = 0, T_{1}v_{i} = v_{i} if i > 1\]

and 
\[T_{2}v_{1} = v_{1}, T_{1}v_{i} = 0 if i > 1\]

They are neither injective, so they are not invertible, and we have 
\[T_{1} + T_{2} = I\]
which is invertible, so the noninvertible operators is not a subspace.

\subsubsection*{3.D.3}
$U$ is a subspace of $V$, so for a basis of $U: v_{1}, \cdots, v_{u}$, we can extend it to a basis of $V: v_{1},\cdots, v_{u}, \cdots, v_{n}$.

If $T$ is invertible and $Tu = Su$, then null $T = {0}$, so $S$ is injective.

If $S$ is injective, then $\forall u \in U, Su \neq {0}$, so $Sv_{1}, \cdots, SV_{u}$ is linear independent. 
Expand the vector to a basis of $V: Sv_{1}, \cdots, Sv_{u}, \cdots, v^{'}_{n}$, and make
\[Tv_{u+1} = v^{'}_{n+1}, \cdots, Tv_{n} = v^{'}_{n}\]

then it's easy to know that T is both injective and surjective, so it's invertible.

\subsubsection*{3.D.4}

If $\exists S, T_{1} = ST_{2}$, then $\forall v\in V, T_{1}v = 0 $ we have
\[T_{1}v = ST_{2}v = 0\]

Since null $T = {0}$, then we have $T_{2}v = 0$ so null $T_{1} = $ null $T_{2}$.

If null $T_{1} = $ null $T_{2}$, then we have $(v_{1},\cdots, v_{n})\, T_{1}v_{i} \neq 0, T_{2}v_{i} \neq 0$.
So we can define $S$ as 
\[S(T_{2}v) = T_{1}v\, v\in (v_{1}, \cdots, v_{n})\]

because $S$ is both injective and surjective.

\subsubsection*{3.D.5}

If $\exists S$, $T_{1}v = (T_{2}S)v = T_{2}(Sv) = $ range $T_{2} v\in V$, so range $T_{1} = $ range $T_{2}$.

If range $T_{1} = $ range $T_{2}$, then dim null $T_{1} = $ dim null $T_{2}$, so we have 2 basis that
\[(v^{'}_{1}, \cdots, v^{'}_{m}, w^{'}_{1}, \cdots, w^{'}_{n})\, T_{1}v = 0, T_{1}w \neq 0\]
and
\[(v^{''}_{1}, \cdots, v^{''}_{m}, w^{''}_{1}, \cdots, w^{''}_{n})\, T_{2}v = 0, T_{2}w \neq 0\]

So we can get an invertible $S$ that
\[Sv^{''} = v^{'}, Sw^{''} = w^{'}\]

\subsubsection*{3.D.6}

If $\exists R, \exists S, T_{1} = ST_{2}R$, then $\forall v\in V\, T_{1}v = (ST_{2}R)v $, so null $T_{1} = $ null $ST_{2}R$.
Because $S,R$ are injective, so dim null $ST_{2}R$ = dim null $T_{2}R = $ dim null $T_{2}$.

If dim null $T_{1} = $ dim null $T_{2}$, then we have 2 basis
\[(v^{'}_{1}, \cdots, v^{'}_{m}, w^{'}_{1}, \cdots, w^{'}_{n})\, T_{1}v = 0, T_{1}w \neq 0\]
and
\[(v^{''}_{1}, \cdots, v^{''}_{m}, w^{''}_{1}, \cdots, w^{''}_{n})\, T_{2}v = 0, T_{2}w \neq 0\]

we make $R$ as $Rv^{'} = Rv^{''}, Rw^{'} = Rv^{''}$ as $S$ as $S(T_{2}v) = T_{1}v$, then we have $T_{1} = ST_{2}R$


\subsubsection*{3.D.7}
\begin{enumerate}[label=(\alph*)]
\item $E$ is subspace
If $T_{1} \in E, T_{2} \in E$, then $(T_{1} + T_{2}v = T_{1}v + T_{2}v = 0 + 0 = 0$. $(\lambda T)v = \lambda (Tv) = \lambda \times 0 = 0$, so $E$ is a subspace.
\item dim $E$
If $v\neq 0$, then we can expand it to a basis of $V: v, v_{2},\cdots, v_{n}$, so $E$ is actually $L($span$(v_{2}, \cdots, v_{n}), W)$, so 
dim $E = ($ dim $V) - 1) \times $ dim $W$.
\end{enumerate}

\subsubsection*{3.D.8}

$V \to W$ is surjective, so suppose null $T = (v_{1}, \cdots, v_{n})$, then we can extend null $T$ to a basis of $V: (v_{1}, \cdots, v_{n}, w_{1},\cdots, w_{m}$.
It's easy to know that $T_{w} \neq 0$, and it's a basis of W.

No define $U = $ span $(w_{1}, \cdots, w_{m}$, and $T|_{U}$ is both injective and surjective, so $T|_{U}$ is isomorphism of $U$ onto $W$.

\subsubsection*{3.D.9}

If $S,T$ are invertible, so 

\[ST\times T^{-1}S^{-1} = S(TT^{-1})S^{-1} = SS^{-1} = I \]

and

\[T^{-1}S^{-1}\times ST = T^{-1}(S^{-1}S)T = T^{-1}T = I \]

so $ST$ is surjective.

IF $ST$ is invertible, then $\forall v, \exists R$ that $R(STv) = I$. If $S$ or $T$ is not inevtible, then $\exists v $ that $STv = 0$, then $Rv = 0$ which is a counter case 
of $R(STv) = I$, so both $S$ and $T$ are invertible.

\subsubsection*{3.D.10}

If $ST = I$, so dim $S = $ dim $V$, so $S$ is surjective: if not so, there would be some $v$ not in range $S$, so $STv \neq v$. $V$ is fixed dimensional, so $S$ is invertible because $3.69$.

\[S^{-1}ST = S^{-1}I = S^{-1} = T\]

so $TS = I$.

\subsubsection*{3.D.11}

Because $3.D.10$, we have $S^{-1} = TU$ and $U^{-1} = ST$.

\begin{equation*}
    \begin{split}
    STU &= I \\
    S^{-1}STU &= S^{-1} \\
    TU &= S^{-1} \\
    TUS &= I
\end{split}
\end{equation*}
and 

\begin{equation*}
    \begin{split}
    STU &= I \\
    USTU &= U \\
    USTUU^{-1} &= UU^{-1} \\
    UST &= I
\end{split}
\end{equation*}
so we have $T^{-1} = US$

\subsubsection*{3.D.13}

Since $V$ is finite dimensional, and $RST$ is surjective, then $RST$ is invertible (3.69); In $3.D.9$, we know that $ST$ is invertable, then $S$ is invertible, so $S$ is injective.

\subsubsection*{3.D.14}

Injective: $\forall v \in V, v \neq 0, \exists a_{1}, \cdots, a_{n}$ that $Tv = [a_{1}, \cdots, a_{n}] \neq {0}$, so $T$ is injective, so $T$ is invertible because $3.69$.

\subsubsection*{3.D.15}

In $3.64$, we have $M(T)_{.,k} = M(v_{k})$, Consider a basis of $V: (v_{1} = [1, 0, \cdots, 0], \cdots, v_{n} = [0, \cdots, 1])$ and a matrix $M$:
\[
\begin{pmatrix}
Tv_{1}, \cdots, Tv_{n}
\end{pmatrix}
\]

for any $T$, It's easy to see that $Mv = Tv$.

\subsubsection*{3.D.20}

The first one equals that

\begin{equation}
    \begin{vmatrix}
    M
    \end{vmatrix} . 
    \begin{bmatrix}
    x_{1} \\ \cdots \\ x_{n}
    \end{bmatrix}  = 0
\end{equation}

If $x_{1} = \cdots = x_{n} = 0$, means $M$ is injective.

The 2nd one equals that

\begin{equation}
    \begin{vmatrix}
    M
    \end{vmatrix} . 
    \begin{bmatrix}
    x_{1} \\ \cdots \\ x_{n}
    \end{bmatrix} = 
    \begin{bmatrix}
    c_{1} \\ \cdots \\ c_{n}
    \end{bmatrix}
\end{equation}

exists a solution, which means that $M$ is invertible, 
they are equivalent because $3.69$.

\subsection*{3.E}
\subsubsection*{3.E.1}

If $T$ is linear, $\forall v_{1}, v_{2} \in T$, we have $T(v_{1} + v_{2}) = T(v_{1}+v_{2})$, which implies 
\[(v_{1}, Tv_{1}) + (v_{2}, Tv_{2}) = (v_{1}+v_{2}, Tv_{1} + Tv_{2}) = (v_{1} + v_{2}, T(v_{1} + v_{2}))\]

using $3.71$. So graph of $T$ is close under addition.

This process also works under scalar multiplication:

\[\lambda Tv = T(\lambda v)\]

so 

\[\lambda (v, Tv) = (\lambda v, \lambda Tv) = (\lambda v, T(\lambda v))\]

so graph of $T$ is linear.

If graph of $T$ is linear, just reverse the process above, we will see that T is linear.

\subsubsection*{3.E.2}

In $3.76$, we have 
\[dim(V_{1} \times \cdots \times V_{m} = dim V_{1} + \cdots + dim V_{m}\] 

so if $V_{1} \times \cdots \times V_{m}$ is finite-dimensional, then each vector space should be finite-dimensional.

\subsubsection*{3.E.3}

Make $U_{1} = P(R), $ and $U_{2} = P(R)$, then 
\[U_{1} \times U_{2} = a, a^{'}, bx, b^{'}x + \cdots \]
and 
\[U_{1} + U_{2} = P(R) = P(R)\]

they are isomorphic because $T$ is invertible if we define $T$ as:
\[T(U_{1} \times U_{2}) = T(a, a^{'}, bx, b^{'}x + \cdots) = a + a^{'}x + bx^{2} + x^{'}x^{3} + \cdots\]

and apparently $U_{1}+U_{2}$ are not direct sum.

\subsubsection*{3.E.4}

$L(V_{1}\times \cdots \times V_{m}, W)$ are maps from $(v_{1}, \cdots, v_{m}) \implies w, w\in W$. Now make $v_{1}, \cdots, v_{m}$ are the only non-zero item in the vector, then the map $S_{0}, \cdots, S_{m}$ would be the same as $T = L(V_{i}, W)$, which means than $\forall T\in L(V, w)$, there is a map equals to it.

For all $S \in L(V_{1}\times \cdots \times V_{m}, W)$, we can split use the previously step, so define
\[R(S) = R(S_{0} + \cdots + S_{m}) = (T_{0} , \cdots , T_{m}) \]

so 

\[S\implies T\] is injective.

Now consider the other direction, 
\[R(T_{0} , \cdots , T_{m}) = S_{0} + \cdots + S_{m} = S\]

$R$ is also injective. Singe the map are injective both direction, so it's a invertible.

\subsubsection*{3.E.5}

This exercise works like $3.E.4$, make $w_{1}, \cdots, w_{m}$ to a vector with only 1 non-zero tiem, then the map equals to $L(V, W_{i})$. so we can find 2 injective mapping with reverse direction.

\subsubsection*{3.E.7}

If $u = 0$, which also belongs to $U$, so $\exists w_{1}$ that 
\[v + 0 = x + w_{1}\]

and for any non-zero $u \in U, \exists w_{2} \in W$ that
\[v + u = x + w_{2}\] 

now use the 2nd equation minus the 1st equation, we got:
\[u = w_{2} - w_{1}\]

Since $W$ is subspace, so $\forall u \in W$, so $U = W$.

\subsubsection*{3.E.8}

If $A$ is a affine subset, then $\exists u, U: u\in V, A = u + U$. $\forall v, w \in A, \lambda \in F, \exists v^{'}, w^{'} \in U$, that

\begin{equation*}
    \begin{split}
    v &= u + v^{'} : v^{'} \in U \\
    w &= u + w^{'} : w^{'} \in U \\
    \lambda v + (1-\lambda)w &= \lambda(u + v^{1}) + (1-\lambda)(u+w^{'}) \\
    &= \lambda u + \lambda v^{'} + (1-\lambda)u + (1-\lambda)w^{'} \\
    &= u + \lambda v^{'} + (1-\lambda)w^{'}
\end{split}
\end{equation*}

Since $U$ is subspace,so $(\lambda v^{'} + (1-\lambda)w^{'}) \in U$, then $(u+\lambda v^{'} + (1-\lambda)w^{'}) \in A$.

To prove that $A$ is affine subset, define $U = {A - v: v\in A, v\neq 0}$, we need to prove that $U$ is a vector space.

$\forall a\in A, (a-v)\in U$, so $\forall \lambda \in F$,

\begin{equation*}
    \begin{split}
    \lambda (a-v) &= \lambda a - \lambda v \\
    &= \lambda a + v - \lambda v -v \\
    & = \lambda a + (1-\lambda)v -v \\
    &= (\lambda a + (1-\lambda)v) - v \in {A-v} = U
\end{split}
\end{equation*}
so $U$ is close under scalar multiplication.

$\forall a, b \in A$, $a-v \in U, b-v\in U$, so $(a-v)/2 \in U, (b-v)/2 \in U$ (by the previously steps).
\begin{equation*}
    \begin{split}
    \frac{a-v}{2} + \frac{b-v}{2} &= \frac{a}{2} + \frac{b}{2} - v \\
    &= \frac{1}{2}a + (1-\frac{1}{2})b -v \\
    &= (\lambda a + (1-\lambda)b) -v \in {A-v} = U
    \end{split}
\end{equation*}

so $U$ is close under addition.

\subsubsection*{3.E.9}

If $A_{1} \cap A_{2} $ is not empty, then there would be at least 1 vector $v$ in both affine subsets, then we have
\[A_{1} = v + U_{1}, A_{2} = v + U_{2}\] 
because $3.85$. Now let's talk about $U_{1}, U_{2}$: 

If $U_{1} \cap U_{2} = {0}$, then $v + {0}$ is a affine subset.

If $U_{1} \cap U_{2} = U_{3} \neq {0}$, then it's easy to know that $U_{3}$ is also a subspace, then

\begin{equation*}
    \begin{split}
    A_{1} \cap A_{2} &= {v + U_{1}} \cap {v + U_{2}} \\
    &= {v + U_{1}\cap U_{2}} \\
    &= {v + U_{3}}
        \end{split}
\end{equation*}

so $A_{1} \cap A_{2} $ is a affine subset or empty.

\subsubsection*{3.E.10}

Use the solution of $3.E.9$, merge 2 affine subsets into a new affine subsets, or a blank set. 

\subsubsection*{3.E.11}

\begin{enumerate}[label=(\alph*)]
\item $A$ is affine subset

like $3.E.8$, we prove that $A-v: v\in A$ is a subspace. It's easy to know that $v_{1} \in A$ because $\lambda_{1} = 1$ will prove that.

\begin{equation*}
    \begin{split}
    k (A-v_{1}) &= k(\lambda_{1} - 1)v_{1} + k\lambda_{2}v_{2} + \cdots + k\lambda_{m}v_{m} \\
    &= k(\lambda_{1} - 1)v_{1} + k\lambda_{2}v_{2} + \cdots + k\lambda_{m}v_{m} + v_{1} - v_{1} \\
    &= (k\lambda_{1} - k + 1)v_{1} + k\lambda_{2}v_{2} + \cdots + k\lambda_{m}v_{m}  - v_{1}
    \end{split}
\end{equation*}

$(k\lambda_{1} - k + 1) + k\lambda_{2} + \cdots + k\lambda_{m} = k\sigma_{1}^{m}\lambda - k + 1 = 1$, 
so $\forall v \in (A-v_{1}), kv \in (A-v_{1})$, which means $A-v_{1}$ is close under multiplication.

The same trick works on addition

\begin{equation*}
    \begin{split}
    (\lambda_{11} - 1)v_{1} + \lambda_{21}v_{2} + \cdots + \lambda_{m1}v_{m} &+ \\
    (\lambda_{12} - 1)v_{1} + \lambda_{22}v_{2} + \cdots + \lambda_{m2}v_{m} &= (\lambda_{11} - 1)v_{1} + \lambda_{21}v_{2} + \cdots + \lambda_{m1}v_{m} +\\
    & (\lambda_{12} - 1)v_{1} + \lambda_{22}v_{2} + \cdots + \lambda_{m2}v_{m} + v_{1} - v_{1} \\
    &= (\lambda_{11} + \lambda_{21} - 1)v_{1} + (\lambda_{21}+\lambda_{22})v_{2} + \cdots + (\lambda_{m1}+\lambda_{m2})v_{m} - v_{1} \\
    \sigma_{1}^{m}\lambda_{1} + \sigma_{1}^{m}\lambda_{2} - 1 &= 1 + 1 - 1 \\
    &= 1
    \end{split}
\end{equation*}

so $A-v_{1}$ is close under addition, Which means $A$ is affine subset.


\item dim $E$
If $v\neq 0$, then we can expand it to a basis of $V: v, v_{2},\cdots, v_{n}$, so $E$ is actually $L($span$(v_{2}, \cdots, v_{n}), W)$, so 
dim $E = ($ dim $V) - 1) \times $ dim $W$.
\end{enumerate}
\newpage
\section{chapter 4}
\section{chapter 5}
\subsection*{5.A}
\subsubsection*{5.A.1}

\begin{enumerate}[label=(\alph*)]
\item Since $U$ is a subspace of $V$, then $0\in U$. $U\subset $ null $T$, so $\forall u\in U, T(u) = 0 \in U $, so $U$ is invariant.
\item $\forall u\in U, T(u)\in $ range $T \subset U$, so $T(u) \in U$, so $U$ is invariant.
\end{enumerate}

\subsubsection*{5.A.2}

$\forall s \in $ null $S, T(s) \in $ null $S$. Prove:

\[S(T(s)) = T(S(s)) = T(0) = 0 \implies T(s) \in null S \].

\subsubsection*{5.A.3}

$\forall s \in $ range $S, \exists s^{'} \in V, S(s^{'}) = s. T(s) = T(S(s^{'})) = S(T(s^{'})) \in $ range $S$, so range $S$ is invariant under $T$.

\subsubsection*{5.A.4}

\[\forall u \in U_{1} + \cdots + U_{m}, u = u_{1} + \cdots + u_{m} \]
\[T(u) = T(u_{1}) + \cdots + T(u_{m}) \in U_{1} + \cdots + U_{m} \]
so $U_{1} + \cdots + U_{m}$ is invariant under $T$.

\subsubsection*{5.A.5}
If not, $\exists v \in $ intersections of $\VEC{V}{n}, T(v) \notin $ intersections of $\VEC{V}{n}$. 
So $\exists V_{i}, T(v) \notin V_{i}$, which is counter case of $V_{i}$ is invariant.

\subsubsection*{5.A.6}
The hypothesis is correct. If $U$ is a subspace of $V$ and $U \neq V, U\neq 0$, then there must a basis of $U \VEC{v}{m}$, and we can extend it to a basis of $V \VEC{v}{n}$. Then $U$ is not invariant under T:

\[
T(u_{i}) =
\begin{cases}
u_{i+1}, & 0<i<n \\
u_{1}, & i= n\\
\end{cases}
\]

because $T(u_{m}) = u_{m+1} \notin U$.

\subsubsection*{5.A.7}

\begin{equation*}
    \begin{split}
    T(x, y) &= \lambda (x, y) \\
    \lambda(x, y) &= (-3y, x) \\
    \lambda x &= -3y \\
    \lambda y &= x \\
    \lambda\lambda y & = -3y \\
    \lambda^{2} &= -3
    \lambda &= \sqrt{3}i and \lambda = -\sqrt{3}i
\end{split}
\end{equation*}

Since $T\in$ L$(R^{2})$, so there is no eigenvalues of $T$.

\subsubsection*{5.A.8}

\begin{equation*}
    \begin{split}
    T(w, z) &= \lambda (w, z) \\
    \lambda(w, z) &= (z, w) \\
    \lambda w &= z \\
    \lambda z &= w \\
    \lambda\lambda w & = w \\
    \lambda^{2} &= 1
    \lambda &= 1 and \lambda = -1
\end{split}
\end{equation*}

The eigenvalues are 1 and -1, the eigenvectors are $(w, w) and (w, -w)$.

\subsubsection*{5.A.9}
\begin{equation*}
    \begin{split}
    T(z_{1}, z_{2}, z_{3}) &= \lambda (z_{1}, z_{2}, z_{3}) \\
    \lambda(z_{1}, z_{2}, z_{3}) &= (2z_{2}, 0, 5z_{3}) \\
    \lambda z_{1} &= 2z_{2} \\
    \lambda z_{2} &= 0 \\
    \lambda z_{3} &= 5z_{3}
\end{split}
\end{equation*}

so $\lambda = 5$ and the eigenvector is $(0, 0, w)$.

\subsubsection*{5.A.10}

\begin{enumerate}[label=(\alph*)]
\item the eigenvalues are $1, 2, \cdots, n$, the eigenvectors are $(w, 0, \cdots, 0), (0, w, 0, \cdots, 0), \cdots, (0, \cdots, 0, w)$.
\item the invariant spaces are $span((\VEC{w}{n})) \exists w_{i} \neq 0$ or ${0}$.
\end{enumerate}

\subsubsection*{5.A.11}
Since the degree of $p^{'}$ is degree $p - 1$, so the eigenvalue is 0, and the eigenvector is $0$.

\subsubsection*{5.A.12}

For $p\in P_{4}, p = a_{4}x^{4} + a_{3}x^{3} + a_{2}x^{2} + a_{1}x + a_{0}$, so 
\begin{equation*}
    \begin{split}
    xp^{'}(x) &= x(4a_{4}x^{3} + 3a_{3}x^{2} + 2a_{2}x + a_{1}) \\
    &= 4a_{4}x^{4} + 3a_{3}x^{3} + 2a_{2}x^{2} + a_{1}x \\
    & = \lambda (a_{4}x^{4} + a_{3}x^{3} + a_{2}x^{2} + a_{1}x + a_{0})
\end{split}
\end{equation*}

so the eigenvalues are $4, 3, 2, 1, 0$ and eigenvectors are $wx^{4}, wx^{3}, wx^{2}, wx, w$

\subsubsection*{5.A.14}

\[P(u+w) = \lambda (u+w) = \lambda u + \lambda w = u\]

the equation works only if $\lambda = 1, w=0$ or $\lambda = 0, u = 0$.
so when $\lambda = 1$, the eigenvectors are $ u\in U, u\neq 0$; when $\lambda = 0$, the eigenvectors are $w \in W, w\neq 0$.

\subsubsection*{5.A.15}
\begin{enumerate}[label=(\alph*)]
\item have the same eigenvalues
\begin{equation*}
    \begin{split}
    T(v) &= \lambda v \\
    T(S(u)) &= \lambda S(u) ; v = S(u) because S is invertible \\
    S^{-1}T(S(u)) &= S^{-1}(\lambda S(u)) \\
    S^{-1}TS(u) &= \lambda S^{-1}S(u) \\
    S^{-1}TS(u) &= \lambda u
\end{split}
\end{equation*}

\item eigenvectors of $T$  are the vectors after transaction (by $S$) of eigenvectors of $S^{-1}TS$.
\end{enumerate}

\subsubsection*{5.A.17}

Consider $T$ as:

\[
\begin{pmatrix}
0 & 0 & 0 & 1 \\
2 & 0 & 0 & 0 \\
0 & 3 & 0 & 0 \\
0 & 0 & 4 & 0
\end{pmatrix}
\]

\subsubsection*{5.A.18}

If $\lambda$ is a eigenvalue of $T$, then we have

\begin{equation*}
    \begin{split}
    T(z_{1}, z_{2}, \cdots) &= \lambda (z_{1}, z_{2}, \cdots) \\
    (0, z_{1}, z_{2}, \cdots) &= \lambda (z_{1}, z_{2}, \cdots) \\
    0 &= \lambda z_{1}, z_{1} = \lambda z_{2}, z_{2} = \lambda z_{3}, \cdots \\
    0 &= z_{1} = z_{2} = \cdots
\end{split}
\end{equation*}
Which is conflict with the define of eigenvalue because in $5.5$, there is a rule  $v\neq 0$.

\subsubsection*{5.A.19}

The eigenvalues are $(0, n)$.
If eigenvalue is n, then eigenvectors are $v = (a, a, \cdots), a \neq 0$; If eigenvalue is 0, then $v = (\VEC{x}{n}), \sum_{i=1}^{n}x_{n} = 0$.

\subsubsection*{5.A.20}

The eigenvalues are $F$.
If eigenvalue is 0, then eigenvectors are $(z, 0, 0, \cdots), z\in F, z \neq 0$; if eigenvalue $\lambda$ is not 0, then eigenvectors are $(z, \lambda z,\lambda^{2}z,\cdots), z\neq 0$.

\subsubsection*{5.A.21}
\begin{enumerate}[label=(\alph*)]
\item \\
suppose $u$ is one eigenvector of $T$, then we have
\begin{equation*}
    \begin{split}
    T(u) &= \lambda u \\
    T^{-1}T(u) &= T^{-1}(\lambda u) \\
    T^{-1}(\lambda u) &= u \\
    T^{-1}(\lambda u) &= \frac{1}{\lambda}\lambda u \\
\end{split}
\end{equation*}

\item 
If $T(u) = \lambda u$, then
\begin{equation*}
    \begin{split}
    u &= T^{-1}T(u) \\
    &= T^{-1}(\lambda u) \\
    \frac{1}{\lambda} &= T^{-1}(\frac{1}{\lambda}\lambda u) \\
    T^{-1}u &= \frac{1}{\lambda}u
\end{split}
\end{equation*}
\end{enumerate}

\subsubsection*{5.A.22}

\begin{equation*}
    \begin{split}
    T(v) &= 3w, T(w) = 3v \\
    T(v + w) &= 3(w+v), T(v - w) = 3(w-v) = -3(v-w)
\end{split}
\end{equation*}

\subsubsection*{5.A.23}

suppose $\lambda$ is an eigenvalue of $ST$, and $v$ is an eigenvector:
\begin{equation*}
    \begin{split}
    T(ST)v &= T(\lambda v) \\
    & = \lambda T(v) \\
    TS(Tv) &= \lambda T(v)
\end{split}
\end{equation*}

so $\lambda$ is also the eigenvalue of $TS$.

\subsubsection*{5.A.24}

\begin{enumerate}[label=(\alph*)]
\item If $v = (i,\cdots, i)^{t}$, then 1 is the eigenvalue of $T$
\item 
\end{enumerate}

\subsubsection*{5.A.25}

If $u, v$ are corresponding to eigenvalues $\lambda_{1}, \lambda_{2}$, suppose $T(u+v) = \lambda_{3}(u+v)$, then we have

\begin{equation*}
    \begin{split}
    T(0) &= 0 \\
    T(u+v - u - v) & = 0\\
    &= T(u+v) - T(u) - T(v) \\
    &= \lambda_{3}(u+v) - \lambda_{1}u - \lambda_{2} v \\
    0 &= (\lambda_{3} - \lambda_{1})u + (\lambda_{3} - \lambda_{2}) v \\
    \lambda_{1} &= \lambda_{2} = \lambda_{3}
\end{split}
\end{equation*}

\subsubsection*{5.A.26}

With the help of $5.A.25$, we know that all vectors have the same eigenvalue, so it's a scalar multiple of the identity operator.

\subsubsection*{5.A.27}

From $5.A.26$, if T is not a scalar multiple of the identify operator, then $\exists u\in V$ and $u$ is not an eigenvector. so $(u, Tu)$ is not linearly independent. Extend it to a basis of $V: (u, Tu, v_{1}, \cdots, v_{n-2})$. Remove $Tu$ from the basis, we can get $U = $ span$(u, v_{1}, \cdots, v_{n-2}$, which is a subspace of $V$ whose dimension is $n-1$. 

Now we have $u\in U$ and $Tu \notin U$, which is a conflict with any subspace with dimension $n-1$ is invariant under $T$.

\subsubsection*{5.A.28}

From $5.A.26$, if T is not scalar multiple of the identify operator, then $\exists u\in V$ and $u_{1}$ is not an eigenvector. Append two linear independent vectors, we got $(u_{1}, u_{2}, u_{3})$.

Since $u_{1}$ is not eigenvector, consider $U$ = span$(u_{1}, u_{2})$, if $Tu_{1} \notin U$, then $U$ is not invariant under $T$; If $Tu_{1} \in U$, then $U_{2}=$span$(u_{1}, u_{3})$ is not invariant under $T$, so $T$ must be a scalar multiple of the identify operator.

\newpage
\section{Some examples to get started}

\subsection{How to create Sections and Subsections}

No introduction.

\subsection{How to include Figures}

First you have to upload the image file from your computer using the upload link in the file-tree menu. Then use the includegraphics command to include it in your document. Use the figure environment and the caption command to add a number and a caption to your figure. See the code for Figure \ref{fig:frog} in this section for an example.

Note that your figure will automatically be placed in the most appropriate place for it, given the surrounding text and taking into account other figures or tables that may be close by. You can find out more about adding images to your documents in this help article on \href{https://www.overleaf.com/learn/how-to/Including_images_on_Overleaf}{including images on Overleaf}.

\begin{figure}
\centering
\includegraphics[width=0.3\textwidth]{frog.jpg}
\caption{\label{fig:frog}This frog was uploaded via the file-tree menu.}
\end{figure}

\subsection{How to add Tables}

Use the table and tabular environments for basic tables --- see Table~\ref{tab:widgets}, for example. For more information, please see this help article on \href{https://www.overleaf.com/learn/latex/tables}{tables}. 

\begin{table}
\centering
\begin{tabular}{l|r}
Item & Quantity \\\hline
Widgets & 42 \\
Gadgets & 13
\end{tabular}
\caption{\label{tab:widgets}An example table.}
\end{table}

\subsection{How to add Comments and Track Changes}

Comments can be added to your project by highlighting some text and clicking ``Add comment'' in the top right of the editor pane. To view existing comments, click on the Review menu in the toolbar above. To reply to a comment, click on the Reply button in the lower right corner of the comment. You can close the Review pane by clicking its name on the toolbar when you're done reviewing for the time being.

Track changes are available on all our \href{https://www.overleaf.com/user/subscription/plans}{premium plans}, and can be toggled on or off using the option at the top of the Review pane. Track changes allow you to keep track of every change made to the document, along with the person making the change. 

\subsection{How to add Lists}

You can make lists with automatic numbering \dots

\begin{enumerate}
\item Like this,
\item and like this.
\end{enumerate}
\dots or bullet points \dots
\begin{itemize}
\item Like this,
\item and like this.
\end{itemize}

\subsection{How to write Mathematics}

\LaTeX{} is great at typesetting mathematics. Let $X_1, X_2, \ldots, X_n$ be a sequence of independent and identically distributed random variables with $\text{E}[X_i] = \mu$ and $\text{Var}[X_i] = \sigma^2 < \infty$, and let
\[S_n = \frac{X_1 + X_2 + \cdots + X_n}{n}
      = \frac{1}{n}\sum_{i}^{n} X_i\]
denote their mean. Then as $n$ approaches infinity, the random variables $\sqrt{n}(S_n - \mu)$ converge in distribution to a normal $\mathcal{N}(0, \sigma^2)$.


\subsection{How to change the margins and paper size}

Usually the template you're using will have the page margins and paper size set correctly for that use-case. For example, if you're using a journal article template provided by the journal publisher, that template will be formatted according to their requirements. In these cases, it's best not to alter the margins directly.

If however you're using a more general template, such as this one, and would like to alter the margins, a common way to do so is via the geometry package. You can find the geometry package loaded in the preamble at the top of this example file, and if you'd like to learn more about how to adjust the settings, please visit this help article on \href{https://www.overleaf.com/learn/latex/page_size_and_margins}{page size and margins}.

\subsection{How to change the document language and spell check settings}

Overleaf supports many different languages, including multiple different languages within one document. 

To configure the document language, simply edit the option provided to the babel package in the preamble at the top of this example project. To learn more about the different options, please visit this help article on \href{https://www.overleaf.com/learn/latex/International_language_support}{international language support}.

To change the spell check language, simply open the Overleaf menu at the top left of the editor window, scroll down to the spell check setting, and adjust accordingly.

\subsection{How to add Citations and a References List}

You can simply upload a \verb|.bib| file containing your BibTeX entries, created with a tool such as JabRef. You can then cite entries from it, like this: \cite{greenwade93}. Just remember to specify a bibliography style, as well as the filename of the \verb|.bib|. You can find a \href{https://www.overleaf.com/help/97-how-to-include-a-bibliography-using-bibtex}{video tutorial here} to learn more about BibTeX.

If you have an \href{https://www.overleaf.com/user/subscription/plans}{upgraded account}, you can also import your Mendeley or Zotero library directly as a \verb|.bib| file, via the upload menu in the file-tree.

\subsection{Good luck!}

We hope you find Overleaf useful, and do take a look at our \href{https://www.overleaf.com/learn}{help library} for more tutorials and user guides! Please also let us know if you have any feedback using the Contact Us link at the bottom of the Overleaf menu --- or use the contact form at \url{https://www.overleaf.com/contact}.

\bibliographystyle{alpha}
\bibliography{sample}

\end{document}