\documentclass{article}

% Language setting
% Replace `english' with e.g. `spanish' to change the document language
\usepackage[english]{babel}

% Set page size and margins
% Replace `letterpaper' with `a4paper' for UK/EU standard size
\usepackage[letterpaper,top=2cm,bottom=2cm,left=3cm,right=3cm,marginparwidth=1.75cm]{geometry}

% Useful packages
\usepackage{amsmath}
\usepackage{graphicx}
\usepackage{enumitem}
\usepackage[colorlinks=true, allcolors=blue]{hyperref}

\title{Linear Algebra Done Right 3rd Edition Homework}
\author{Liszt Li}

\begin{document}
\maketitle

\begin{abstract}
The homework of Linear Algebra Done Right 3rd Edition.
\end{abstract}

\section*{Introduction}
No introduction
\section{chapter 1}
\subsection*{1.A}
\subsubsection*{1.A.1}

\[\frac{1}{(a+bi)} = c + di \]

\begin{equation*}
\begin{split}
    \frac{1\cdot (a-bi)}{(a+bi)(a-bi)} & = \frac{a-bi}{a^2+b^2} \\
& = \frac{a}{a^2+b^2}+ \frac{-b}{a^2+b^2}i \\
\end{split}
\end{equation*}

\[c=\frac{a}{a^2+b^2} ,d=\frac{-b}{a^2+b^2}\]

\subsubsection*{1.A.2}
\begin{equation*}
\begin{split}
    \left( \frac{-1+\sqrt{3}i}{2}\right ) ^3 &= \frac{(-1+\sqrt{3}i)^3}{8} \\
    &= \frac{(-1)^3+3\cdot (-1)^2 \cdot \sqrt{3}i + 3 \cdot -1 \cdot (\sqrt{3}i)^2 + (\sqrt{3}i)^3}{8} \\
    & = \frac{-1 + 3\sqrt{3}i + 9 + (-3\sqrt{3}i)}{8} = \frac{8}{8} \\
    & = 1
\end{split}
\end{equation*}

\subsubsection*{1.A.3}

\begin{equation*}
    \begin{split}
        i = (a+bi)^2 & = (a^2 + 2abi - b^2) \\
        & = (a^2 - b^2) + 2ab\cdot i \\
        & = 0 + i
    \end{split}
\end{equation*}
so we have:
\[a=\pm b, a\cdot b = \frac{1}{2} \]
\[a= \pm \sqrt{\frac{1}{2}}\]
then:
\[\sqrt{i} = \sqrt{\frac{1}{2}} + \sqrt{\frac{1}{2}}i\]
or
\[\sqrt{i} = -\sqrt{\frac{1}{2}} - \sqrt{\frac{1}{2}}i\]

\subsubsection*{1.A.4}
\begin{equation*}
    \begin{split}
C & ={a+bi:a,b \in R}. \\
\alpha + \beta & = a_\alpha +b_\beta i + b_\alpha + b_\beta i \\
& = b_\alpha + b_\beta i + a_\alpha +b_\beta i \\
& = \beta + \alpha
    \end{split}
\end{equation*}

\subsubsection*{1.A.10}
\begin{equation*}
    \begin{split}
        (4, -3, 1, 7) + 2x &= (5, 9, -6, 8) \\
        \implies  
        2x & = (5, 9, -6, 8) - (4, -3, 1, 7) \\
        & = (1, 12, -7, 1) \\
        \implies
        x &= (\frac{1}{2}, 6, -\frac{7}{2}, \frac{1}{2})
    \end{split}
\end{equation*}

\subsubsection*{1.A.11}
\begin{equation*}
\frac{12-5i}{2-3i} = 3+2i 
\implies
(-6+7i) \cdot (2-3i) = (-32+9i) \neq (-32-9i)
\end{equation*}

\subsection*{1.B}

\subsubsection*{1.B.1}
\begin{equation*}
    (-v) + -(-v) = 0 = -v + v \implies -(-v) = v
\end{equation*}

\subsubsection*{1.B.2}
\begin{equation*}
    a\cdot v = (av_0, av_1, \cdots av_n) = (0, 0, \cdots 0) \\
    \implies a = 0\, or\, v = (0, 0, \cdots 0)
\end{equation*}

\subsubsection*{1.B.3}
\begin{equation*}
    x = (\frac{w_1 - v_1}{3}, \frac{w_2-v_2}{3}, \cdots \frac{w_n-v_n}{3})
\end{equation*}

\subsubsection*{1.B.4}
additive inverse

\subsubsection*{1.B.5}
we have
\begin{equation*}
    0v = 0, 1v = v \implies    (0-1)v = 0 - v \implies -v + v = 0
\end{equation*}

\subsubsection*{1.B.6}
In 1.19, associativity we have $(u+v)+w = u+(v+w)$ if it's a vector space.

So If $R \cup \infty \cup -\infty$ is a vector space, we can calculate
    \[ (1 + \infty) + -\infty = \infty + -\infty = 0 \]
    \[1 + (\infty + -\infty) = 1 + 0 = 1\]
which breaks the associativity rule, so it's not a vector space.


\subsection*{1.C}

\subsubsection*{1.C.1}
\begin{enumerate}[label=(\alph*)]
\item Yes
\item No, $(0, 0, 0)$ not in subset
\item Yes
\item Yes
\end{enumerate}

\subsubsection*{1.C.3}
In 1.34, three are 3 conditions:

additive identity:

$ f \equiv 1$

close under addition

$(f+g)^{'} (-1) = f^{'} (-1) + g^{'} (-1) = 3f(2) + 3g(2) = 3(f+g)(2) $

close under scalar multiplication

$(u\cdot f)^{'}(-1) = u\cdot f^{'}(-1) = u\cdot 3f(2) = 3(u\cdot f)(2)$

\subsubsection*{1.C.4}

ONLY IF it is a subspace, then we have

\[ \int_{0}^{1}(uf) = u\int_{0}^{1}(f) = u\cdot b = b \]

so b must be 0 since u is not zero.


IF

additive identity:

$ f \equiv 1$

close under addition

$\int_{0}^{1}(f+g) = \int_{0}^{1}(f) +  \int_{0}^{1}(f) = 0 + 0 = 0 $

close under scalar multiplication

See the "only if" part

\subsubsection*{1.C.5}

No

\[ u = (1, 1) \in R^{2}, a= i \in C \]


\[ a\cdot u = (i, i) \notin R^{2} \]

\subsubsection*{1.C.6}

\begin{enumerate}[label=(\alph*)]
\item Yes

$a\in R, a^{3}=b^{3} \implies a = b $ so $(a, a, c)$ is a subspace of $R^{3}$.
\item No

we can use polar coordinate to explain why. When we calculate $\times$ in polar coordinate, it equals to extend the length, and rotate the degree anti-clock wise. So all these coordinate has the same square results:
\[(\rho, \theta + 120\cdot n) \,n \in R\]

so the following two vectors will break the "close under addition" rule:
\[((1, 90), (1, 90), (1, 0))\]
\[((1, 90), (1, 210), (1, 0))\]

because the length after sum will not be equal:  $(1, 90) + (1, 90)$ has length 2 but $(1, 90) + (1, 210)$ is shorter than 2(they are not parallel).
\end{enumerate}

\subsubsection*{1.C.7}

Make $U=\{(x, y): x, y\in N\}$, then $(0.5\cdot u) \notin U $.

\subsubsection*{1.C.8}

Make $U=\{(x, y): x\cdot y \geq 0\}$, then $(-5,0) \in U, (0,1) \in U$, but $(-5, 1) \notin U$.

\subsubsection*{1.C.9}

Clearly periodic functions meets additive identity and close under scalar multiplication. 

For addition we have 
\[(f+g)(x) = (f+g)(x+lcm(p))\]
so it's close to addition.

\subsubsection*{1.C.10}
$0 \in U_{1} \cap U_{2}$

if $ u, v \in U_{1} \cap U_{2}$, then $u+v in U_{1}$ and $U_{2}$, so $u+v \in U_{1} \cap U_{2}$

if $ u \in U_{1} \cap U_{2}$, then $a\cdot u in U_{1}$ and $U_{2}$, so $a\cdot u \in U_{1} \cap U_{2}$

\subsubsection*{1.C.11}

Just reduce 1.C.10 to all the subsets of V, and the conclusion still works.

\subsubsection*{1.C.12}

ONLY IF

Assume we have $U_{1}, U_{2}$ are two subsets of V, and make
\[U_{1} - U_{2} = A, U_{2} - U_{1} = C\]
if Both $A$ and $C$ are not blank, then we pick any $u_{a}, u_{c}$ and make $u_{a} + u_{c} = u_{ac}$ which is also in $U_{1}$ (it also works if $u_{ac}$ is in $U_{2}$, just switch $U_{1}, U_{2}$).

Since $u_{ac} and u_{a}$ are both in $U_{1}$, we have
\[u_{ac} + -u_{a} \in U_{1}\]
\[u_{a} + u_{c} + -u_{a} \in U_{1}\]
\[u_{c} \in U_{1}\]
Which is conflict with our assumuption.

\subsubsection*{1.C.15}

For any $u_{1}, u_{2} \in U$, we have $u_{1} + u_{2} \in U$, so $U+U=U$.

\subsubsection*{1.C.16}
\[U + W = \{u + w:u \in U, w \in W\} = \{w + u: w\in W, u\in U\} = W + U\]

\subsubsection*{1.C.17}
\[(U_{1} + U_{2}) + U_{3} = \{(u_{1} + u_{2}) + u_{3}: u_{1} \in U_{1}, u_{2} \in U_{2}, u_{3} \in U_{3}\} = \]

\[= \{u_{1} + (u_{2} + u_{3}): u_{1} \in U_{1}, u_{2} \in U_{2}, u_{3} \in U_{3}\} = U_{1} + (U_{2} + U_{3})\]

\subsubsection*{1.C.19}
\[U_{1} = \{(x, 0) \in R^{2}\}\]
\[U_{2} = \{(x, y) \in R^{2}\}\]
\[W = \{(0, y) \in R^{2}\}\]

\[U_{1} \neq U_{2}, U_{1}+W = U_{2}+W\].

\subsubsection*{1.C.20}

\[W = \{(0, w, 0, z) \in F^{4}: w, z\in F\}\]

\subsubsection*{1.C.21}
\[W = \{(0, 0, v, w, z) \in F^{5}: v,w,z \in F\}\]

\subsubsection*{1.C.22}
\[W_{1} = \{(0, 0, v, 0,0) \in F^{5}: v \in F\} \]
\[W_{2} = \{(0, 0, 0, w, 0) \in F^{5}: w \in F\}\]
\[W_{3} = \{(0, 0, 0, 0, z) \in F^{5}: z \in F\}\]

\subsubsection*{1.C.23}

The case in $1.C.19$ also works in this exercise.


\newpage

\section{Some examples to get started}

\subsection{How to create Sections and Subsections}

No introduction.

\subsection{How to include Figures}

First you have to upload the image file from your computer using the upload link in the file-tree menu. Then use the includegraphics command to include it in your document. Use the figure environment and the caption command to add a number and a caption to your figure. See the code for Figure \ref{fig:frog} in this section for an example.

Note that your figure will automatically be placed in the most appropriate place for it, given the surrounding text and taking into account other figures or tables that may be close by. You can find out more about adding images to your documents in this help article on \href{https://www.overleaf.com/learn/how-to/Including_images_on_Overleaf}{including images on Overleaf}.

\begin{figure}
\centering
\includegraphics[width=0.3\textwidth]{frog.jpg}
\caption{\label{fig:frog}This frog was uploaded via the file-tree menu.}
\end{figure}

\subsection{How to add Tables}

Use the table and tabular environments for basic tables --- see Table~\ref{tab:widgets}, for example. For more information, please see this help article on \href{https://www.overleaf.com/learn/latex/tables}{tables}. 

\begin{table}
\centering
\begin{tabular}{l|r}
Item & Quantity \\\hline
Widgets & 42 \\
Gadgets & 13
\end{tabular}
\caption{\label{tab:widgets}An example table.}
\end{table}

\subsection{How to add Comments and Track Changes}

Comments can be added to your project by highlighting some text and clicking ``Add comment'' in the top right of the editor pane. To view existing comments, click on the Review menu in the toolbar above. To reply to a comment, click on the Reply button in the lower right corner of the comment. You can close the Review pane by clicking its name on the toolbar when you're done reviewing for the time being.

Track changes are available on all our \href{https://www.overleaf.com/user/subscription/plans}{premium plans}, and can be toggled on or off using the option at the top of the Review pane. Track changes allow you to keep track of every change made to the document, along with the person making the change. 

\subsection{How to add Lists}

You can make lists with automatic numbering \dots

\begin{enumerate}
\item Like this,
\item and like this.
\end{enumerate}
\dots or bullet points \dots
\begin{itemize}
\item Like this,
\item and like this.
\end{itemize}

\subsection{How to write Mathematics}

\LaTeX{} is great at typesetting mathematics. Let $X_1, X_2, \ldots, X_n$ be a sequence of independent and identically distributed random variables with $\text{E}[X_i] = \mu$ and $\text{Var}[X_i] = \sigma^2 < \infty$, and let
\[S_n = \frac{X_1 + X_2 + \cdots + X_n}{n}
      = \frac{1}{n}\sum_{i}^{n} X_i\]
denote their mean. Then as $n$ approaches infinity, the random variables $\sqrt{n}(S_n - \mu)$ converge in distribution to a normal $\mathcal{N}(0, \sigma^2)$.


\subsection{How to change the margins and paper size}

Usually the template you're using will have the page margins and paper size set correctly for that use-case. For example, if you're using a journal article template provided by the journal publisher, that template will be formatted according to their requirements. In these cases, it's best not to alter the margins directly.

If however you're using a more general template, such as this one, and would like to alter the margins, a common way to do so is via the geometry package. You can find the geometry package loaded in the preamble at the top of this example file, and if you'd like to learn more about how to adjust the settings, please visit this help article on \href{https://www.overleaf.com/learn/latex/page_size_and_margins}{page size and margins}.

\subsection{How to change the document language and spell check settings}

Overleaf supports many different languages, including multiple different languages within one document. 

To configure the document language, simply edit the option provided to the babel package in the preamble at the top of this example project. To learn more about the different options, please visit this help article on \href{https://www.overleaf.com/learn/latex/International_language_support}{international language support}.

To change the spell check language, simply open the Overleaf menu at the top left of the editor window, scroll down to the spell check setting, and adjust accordingly.

\subsection{How to add Citations and a References List}

You can simply upload a \verb|.bib| file containing your BibTeX entries, created with a tool such as JabRef. You can then cite entries from it, like this: \cite{greenwade93}. Just remember to specify a bibliography style, as well as the filename of the \verb|.bib|. You can find a \href{https://www.overleaf.com/help/97-how-to-include-a-bibliography-using-bibtex}{video tutorial here} to learn more about BibTeX.

If you have an \href{https://www.overleaf.com/user/subscription/plans}{upgraded account}, you can also import your Mendeley or Zotero library directly as a \verb|.bib| file, via the upload menu in the file-tree.

\subsection{Good luck!}

We hope you find Overleaf useful, and do take a look at our \href{https://www.overleaf.com/learn}{help library} for more tutorials and user guides! Please also let us know if you have any feedback using the Contact Us link at the bottom of the Overleaf menu --- or use the contact form at \url{https://www.overleaf.com/contact}.

\bibliographystyle{alpha}
\bibliography{sample}

\end{document}