\section{chapter 1}
\subsection*{1.A}
\subsubsection*{1.A.1}

\[\frac{1}{(a+bi)} = c + di \]

\begin{equation*}
\begin{split}
    \frac{1\cdot (a-bi)}{(a+bi)(a-bi)} & = \frac{a-bi}{a^2+b^2} \\
& = \frac{a}{a^2+b^2}+ \frac{-b}{a^2+b^2}i \\
\end{split}
\end{equation*}

\[c=\frac{a}{a^2+b^2} ,d=\frac{-b}{a^2+b^2}\]

\subsubsection*{1.A.2}
\begin{equation*}
\begin{split}
    \left( \frac{-1+\sqrt{3}i}{2}\right ) ^3 &= \frac{(-1+\sqrt{3}i)^3}{8} \\
    &= \frac{(-1)^3+3\cdot (-1)^2 \cdot \sqrt{3}i + 3 \cdot -1 \cdot (\sqrt{3}i)^2 + (\sqrt{3}i)^3}{8} \\
    & = \frac{-1 + 3\sqrt{3}i + 9 + (-3\sqrt{3}i)}{8} = \frac{8}{8} \\
    & = 1
\end{split}
\end{equation*}

\subsubsection*{1.A.3}

\begin{equation*}
    \begin{split}
        i = (a+bi)^2 & = (a^2 + 2abi - b^2) \\
        & = (a^2 - b^2) + 2ab\cdot i \\
        & = 0 + i
    \end{split}
\end{equation*}
so we have:
\[a=\pm b, a\cdot b = \frac{1}{2} \]
\[a= \pm \sqrt{\frac{1}{2}}\]
then:
\[\sqrt{i} = \sqrt{\frac{1}{2}} + \sqrt{\frac{1}{2}}i\]
or
\[\sqrt{i} = -\sqrt{\frac{1}{2}} - \sqrt{\frac{1}{2}}i\]

\subsubsection*{1.A.4}
\begin{equation*}
    \begin{split}
C & ={a+bi:a,b \in R}. \\
\alpha + \beta & = a_\alpha +b_\beta i + b_\alpha + b_\beta i \\
& = b_\alpha + b_\beta i + a_\alpha +b_\beta i \\
& = \beta + \alpha
    \end{split}
\end{equation*}

\subsubsection*{1.A.10}
\begin{equation*}
    \begin{split}
        (4, -3, 1, 7) + 2x &= (5, 9, -6, 8) \\
        \implies  
        2x & = (5, 9, -6, 8) - (4, -3, 1, 7) \\
        & = (1, 12, -7, 1) \\
        \implies
        x &= (\frac{1}{2}, 6, -\frac{7}{2}, \frac{1}{2})
    \end{split}
\end{equation*}

\subsubsection*{1.A.11}
\begin{equation*}
\frac{12-5i}{2-3i} = 3+2i 
\implies
(-6+7i) \cdot (2-3i) = (-32+9i) \neq (-32-9i)
\end{equation*}

\subsection*{1.B}

\subsubsection*{1.B.1}
\begin{equation*}
    (-v) + -(-v) = 0 = -v + v \implies -(-v) = v
\end{equation*}

\subsubsection*{1.B.2}
\begin{equation*}
    a\cdot v = (av_0, av_1, \cdots av_n) = (0, 0, \cdots 0) \\
    \implies a = 0\, or\, v = (0, 0, \cdots 0)
\end{equation*}

\subsubsection*{1.B.3}
\begin{equation*}
    x = (\frac{w_1 - v_1}{3}, \frac{w_2-v_2}{3}, \cdots \frac{w_n-v_n}{3})
\end{equation*}

\subsubsection*{1.B.4}
additive inverse

\subsubsection*{1.B.5}
we have
\begin{equation*}
    0v = 0, 1v = v \implies    (0-1)v = 0 - v \implies -v + v = 0
\end{equation*}

\subsubsection*{1.B.6}
In 1.19, associativity we have $(u+v)+w = u+(v+w)$ if it's a vector space.

So If $R \cup \infty \cup -\infty$ is a vector space, we can calculate
    \[ (1 + \infty) + -\infty = \infty + -\infty = 0 \]
    \[1 + (\infty + -\infty) = 1 + 0 = 1\]
which breaks the associativity rule, so it's not a vector space.


\subsection*{1.C}

\subsubsection*{1.C.1}
\begin{enumerate}[label=(\alph*)]
\item Yes
\item No, $(0, 0, 0)$ not in subset
\item Yes
\item Yes
\end{enumerate}

\newpage