\section{chapter 1}
\subsection*{1.A}
\subsubsection*{1.A.1}

\[\frac{1}{(a+bi)} = c + di \]

\begin{equation*}
\begin{split}
    \frac{1\cdot (a-bi)}{(a+bi)(a-bi)} & = \frac{a-bi}{a^2+b^2} \\
& = \frac{a}{a^2+b^2}+ \frac{-b}{a^2+b^2}i \\
\end{split}
\end{equation*}

\[c=\frac{a}{a^2+b^2} ,d=\frac{-b}{a^2+b^2}\]

\subsubsection*{1.A.2}
\begin{equation*}
\begin{split}
    \left( \frac{-1+\sqrt{3}i}{2}\right ) ^3 &= \frac{(-1+\sqrt{3}i)^3}{8} \\
    &= \frac{(-1)^3+3\cdot (-1)^2 \cdot \sqrt{3}i + 3 \cdot -1 \cdot (\sqrt{3}i)^2 + (\sqrt{3}i)^3}{8} \\
    & = \frac{-1 + 3\sqrt{3}i + 9 + (-3\sqrt{3}i)}{8} = \frac{8}{8} \\
    & = 1
\end{split}
\end{equation*}

\subsubsection*{1.A.3}

\begin{equation*}
    \begin{split}
        i = (a+bi)^2 & = (a^2 + 2abi - b^2) \\
        & = (a^2 - b^2) + 2ab\cdot i \\
        & = 0 + i
    \end{split}
\end{equation*}
so we have:
\[a=\pm b, a\cdot b = \frac{1}{2} \]
\[a= \pm \sqrt{\frac{1}{2}}\]
then:
\[\sqrt{i} = \sqrt{\frac{1}{2}} + \sqrt{\frac{1}{2}}i\]
or
\[\sqrt{i} = -\sqrt{\frac{1}{2}} - \sqrt{\frac{1}{2}}i\]

\subsubsection*{1.A.4}
\begin{equation*}
    \begin{split}
C & ={a+bi:a,b \in R}. \\
\alpha + \beta & = a_\alpha +b_\beta i + b_\alpha + b_\beta i \\
& = b_\alpha + b_\beta i + a_\alpha +b_\beta i \\
& = \beta + \alpha
    \end{split}
\end{equation*}

\subsubsection*{1.A.10}
\begin{equation*}
    \begin{split}
        (4, -3, 1, 7) + 2x &= (5, 9, -6, 8) \\
        \implies  
        2x & = (5, 9, -6, 8) - (4, -3, 1, 7) \\
        & = (1, 12, -7, 1) \\
        \implies
        x &= (\frac{1}{2}, 6, -\frac{7}{2}, \frac{1}{2})
    \end{split}
\end{equation*}

\subsubsection*{1.A.11}
\begin{equation*}
\frac{12-5i}{2-3i} = 3+2i 
\implies
(-6+7i) \cdot (2-3i) = (-32+9i) \neq (-32-9i)
\end{equation*}

\subsection*{1.B}

\subsubsection*{1.B.1}
\begin{equation*}
    (-v) + -(-v) = 0 = -v + v \implies -(-v) = v
\end{equation*}

\subsubsection*{1.B.2}
\begin{equation*}
    a\cdot v = (av_0, av_1, \cdots av_n) = (0, 0, \cdots 0) \\
    \implies a = 0\, or\, v = (0, 0, \cdots 0)
\end{equation*}

\subsubsection*{1.B.3}
\begin{equation*}
    x = (\frac{w_1 - v_1}{3}, \frac{w_2-v_2}{3}, \cdots \frac{w_n-v_n}{3})
\end{equation*}

\subsubsection*{1.B.4}
additive inverse

\subsubsection*{1.B.5}
we have
\begin{equation*}
    0v = 0, 1v = v \implies    (0-1)v = 0 - v \implies -v + v = 0
\end{equation*}

\subsubsection*{1.B.6}
In 1.19, associativity we have $(u+v)+w = u+(v+w)$ if it's a vector space.

So If $R \cup \infty \cup -\infty$ is a vector space, we can calculate
    \[ (1 + \infty) + -\infty = \infty + -\infty = 0 \]
    \[1 + (\infty + -\infty) = 1 + 0 = 1\]
which breaks the associativity rule, so it's not a vector space.


\subsection*{1.C}

\subsubsection*{1.C.1}
\begin{enumerate}[label=(\alph*)]
\item Yes
\item No, $(0, 0, 0)$ not in subset
\item Yes
\item Yes
\end{enumerate}

\subsubsection*{1.C.3}
In 1.34, three are 3 conditions:

additive identity:

$ f \equiv 1$

close under addition

$(f+g)^{'} (-1) = f^{'} (-1) + g^{'} (-1) = 3f(2) + 3g(2) = 3(f+g)(2) $

close under scalar multiplication

$(u\cdot f)^{'}(-1) = u\cdot f^{'}(-1) = u\cdot 3f(2) = 3(u\cdot f)(2)$

\subsubsection*{1.C.4}

ONLY IF it is a subspace, then we have

\[ \int_{0}^{1}(uf) = u\int_{0}^{1}(f) = u\cdot b = b \]

so b must be 0 since u is not zero.


IF

additive identity:

$ f \equiv 1$

close under addition

$\int_{0}^{1}(f+g) = \int_{0}^{1}(f) +  \int_{0}^{1}(f) = 0 + 0 = 0 $

close under scalar multiplication

See the "only if" part

\subsubsection*{1.C.5}

No

\[ u = (1, 1) \in R^{2}, a= i \in C \]


\[ a\cdot u = (i, i) \notin R^{2} \]

\subsubsection*{1.C.6}

\begin{enumerate}[label=(\alph*)]
\item Yes

$a\in R, a^{3}=b^{3} \implies a = b $ so $(a, a, c)$ is a subspace of $R^{3}$.
\item No

we can use polar coordinate to explain why. When we calculate $\times$ in polar coordinate, it equals to extend the length, and rotate the degree anti-clock wise. So all these coordinate has the same square results:
\[(\rho, \theta + 120\cdot n) \,n \in R\]

so the following two vectors will break the "close under addition" rule:
\[((1, 90), (1, 90), (1, 0))\]
\[((1, 90), (1, 210), (1, 0))\]

because the length after sum will not be equal:  $(1, 90) + (1, 90)$ has length 2 but $(1, 90) + (1, 210)$ is shorter than 2(they are not parallel).
\end{enumerate}

\subsubsection*{1.C.7}

Make $U=\{(x, y): x, y\in N\}$, then $(0.5\cdot u) \notin U $.

\subsubsection*{1.C.8}

Make $U=\{(x, y): x\cdot y \geq 0\}$, then $(-5,0) \in U, (0,1) \in U$, but $(-5, 1) \notin U$.

\subsubsection*{1.C.9}

Clearly periodic functions meets additive identity and close under scalar multiplication. 

For addition we have 
\[(f+g)(x) = (f+g)(x+lcm(p))\]
so it's close to addition.

\subsubsection*{1.C.10}
$0 \in U_{1} \cap U_{2}$

if $ u, v \in U_{1} \cap U_{2}$, then $u+v in U_{1}$ and $U_{2}$, so $u+v \in U_{1} \cap U_{2}$

if $ u \in U_{1} \cap U_{2}$, then $a\cdot u in U_{1}$ and $U_{2}$, so $a\cdot u \in U_{1} \cap U_{2}$

\subsubsection*{1.C.11}

Just reduce 1.C.10 to all the subsets of V, and the conclusion still works.

\subsubsection*{1.C.12}

ONLY IF

Assume we have $U_{1}, U_{2}$ are two subsets of V, and make
\[U_{1} - U_{2} = A, U_{2} - U_{1} = C\]
if Both $A$ and $C$ are not blank, then we pick any $u_{a}, u_{c}$ and make $u_{a} + u_{c} = u_{ac}$ which is also in $U_{1}$ (it also works if $u_{ac}$ is in $U_{2}$, just switch $U_{1}, U_{2}$).

Since $u_{ac} and u_{a}$ are both in $U_{1}$, we have
\[u_{ac} + -u_{a} \in U_{1}\]
\[u_{a} + u_{c} + -u_{a} \in U_{1}\]
\[u_{c} \in U_{1}\]
Which is conflict with our assumuption.

\subsubsection*{1.C.15}

For any $u_{1}, u_{2} \in U$, we have $u_{1} + u_{2} \in U$, so $U+U=U$.

\subsubsection*{1.C.16}
\[U + W = \{u + w:u \in U, w \in W\} = \{w + u: w\in W, u\in U\} = W + U\]

\subsubsection*{1.C.17}
\[(U_{1} + U_{2}) + U_{3} = \{(u_{1} + u_{2}) + u_{3}: u_{1} \in U_{1}, u_{2} \in U_{2}, u_{3} \in U_{3}\} = \]

\[= \{u_{1} + (u_{2} + u_{3}): u_{1} \in U_{1}, u_{2} \in U_{2}, u_{3} \in U_{3}\} = U_{1} + (U_{2} + U_{3})\]

\subsubsection*{1.C.19}
\[U_{1} = \{(x, 0) \in R^{2}\}\]
\[U_{2} = \{(x, y) \in R^{2}\}\]
\[W = \{(0, y) \in R^{2}\}\]

\[U_{1} \neq U_{2}, U_{1}+W = U_{2}+W\].

\subsubsection*{1.C.20}

\[W = \{(0, w, 0, z) \in F^{4}: w, z\in F\}\]

\subsubsection*{1.C.21}
\[W = \{(0, 0, v, w, z) \in F^{5}: v,w,z \in F\}\]

\subsubsection*{1.C.22}
\[W_{1} = \{(0, 0, v, 0,0) \in F^{5}: v \in F\} \]
\[W_{2} = \{(0, 0, 0, w, 0) \in F^{5}: w \in F\}\]
\[W_{3} = \{(0, 0, 0, 0, z) \in F^{5}: z \in F\}\]

\subsubsection*{1.C.23}

The case in $1.C.19$ also works in this exercise.


\newpage