\section{chapter 2}
\subsection*{2.A}
\subsubsection*{2.A.1}

\begin{equation*}
    \begin{split}
w_{1} &= v_{1} - v_{2}, w_{2} = v_{2} - v_{3}, w_{3} = v_{3} - v_{4}, w_{4} = v_{4} \\
&\implies \\
v_{1} &= w_{1} + w_{2} + w{3} + w{4} \\
v_{2} &= w_{2} + w{3} + w{4} \\
v_{3} &= w{3} + w{4} \\
v_{4} &= w{4} \\
    \end{split}
\end{equation*}

\[
\forall a_{1}, a_{2}, a_{3}, a_{4} \in F \,
a_{1}v_{1} + a_{2}v_{2} + a_{3}v_{3} + a_{4}v_{4} = 
a_{1}w_{1} + (a_{1} + a_{2})w_{2}  + (a_{1} + a_{2} +a_{3})w_{3} + (a_{1} + a_{2} +a_{3}+a_{4})w_{4}
\]

\subsubsection*{2.A.3}
\[3a+2b = 5, a - 3b = 9 \implies a=3, b=-2 \implies t = 3\cdot 4 + (-2)\cdot 5 = 4 \]

\subsubsection*{2.A.5}
\begin{enumerate}[label=(\alph*)]
\item $1+i, 1-i$ point different direction in polar coordinate, and times $a\in R$ does not change the direction, so they are independent over $R$
\item We can build a counterexample

\[(1-i)(1+i) = (1+i)(1-i) = -(-1-i)(1-i) \implies (1-i)(1+i) + (-1-i)(1-i) = 0\]
\end{enumerate}

\subsubsection*{2.A.6}

if $v_{1}-v_{2}, v_{2} - v_{3}, v_{3}-v_{4}, v_{4}$ is linearly dependent, then 
\[\exists\, a_{1}(v_{1}-v_{2})+ a_{2}(v_{2} - v{3})+ a_{3}(v{3}-v{4})+ a_{4}v_{4} = 0 \]

\[\implies a_{1}v_{1} + (a_{2} - a_{1})v_{2} + (a_{3}-a_{2})v_{3} + (a_{4}-a_{3})v_{4} = 0\]

which is a counter example than $v_{1}, v_{2}, v_{3}, v_{4}$ is linearly independent

\subsubsection*{2.A.7}

IF $\exists a_{1}, a_{2}, \cdots,a_{m}$ that makes
\[a_{1}(5v_{1}-4v_{2}) + a_{2}v_{2}+\cdots + a_{m}v_{m} = 0\]
Then we have
\[5a_{1}v_{1} + (a_{2}-4a_{1})v_{2}+\cdots + a_{m}v_{m} = 0\]

which is a counter example.

\subsubsection*{2.A.8}

IF $\lambda v_{1}, \lambda v_{2}, \cdots, \lambda v_{m}$ is linearly dependent, then 
\[a_{1}\lambda v_{1} + a_{2}\lambda v_{2} + \cdots + a_{m}\lambda v_{m} = 0\]
\[(a_{1}\lambda v_{1}) + (a_{2}\lambda) v_{2} + \cdots + (a_{m}\lambda v_{m}) = 0\]
which is a counterexample that $v_{1}, v_{2}, \cdots, v_{m}$ is linearly independent.

\subsubsection*{2.A.9}

Counterexample:

\[w_{1} = -v_{1}, w_{2} = -v_{2},\cdots, w_{m} = -v_{m}\]

\subsubsection*{2.A.10}

\[a_{1}(v_{1}+w) + a_{2}(v_{2} + w)+\cdots + a_{m}(v_{m} + w) = 0\]
\[a_{1}v_{1} + a_{2}v_{2} + \cdots +a_{m}v_{m} + (a_{1}+a_{2}+\cdots +a_{m})w = 0\]
\[w=-\frac{a_{1}}{a_{1}+a_{2}+\cdots +a_{m}}v_{1} - \frac{a_{2}}{a_{1}+a_{2}+\cdots +a_{m}}v_{2} - \cdots - \frac{a_{m}}{a_{1}+a_{2}+\cdots +a_{m}}v_{m} \]

so $w\in span(v_{1},\cdots ,v_{m}$

\subsubsection*{2.A.11}

Use the process in 2.A.10 to get counterexample.

\subsubsection*{2.A.12}

$(1, 0, 0, 0, 0), (0, z, 0,0,0), (0,0,z^{2},0,0,),(0,0,0,z^{3},0),(0,0,0,0,z^{4})$ spans $P_{4}(F)$, so the length of spanning list is 5; if six polynomials is linearly independent, it's conflict with 2.23: "Length of linearly independent list $\leq$ length of spanning list"

\subsubsection*{2.A.13}

also we can leverage 2.23. It's easy to see that
\[(1, 0, 0, 0, 0), (0, z, 0,0,0), (0,0,z^{2},0,0,),(0,0,0,z^{3},0),(0,0,0,0,z^{4})\]
is linearly independent, and it has length 5; if 4 polynomials spans $P_{4}(F)$, it's conflict with 2.23

\subsubsection*{2.A.14}
\begin{enumerate}[label=(\alph*)]
\item IF:

$\forall m$, $\exists $ linearly independent vectors in $V$, so there is no number $m$ than make $m$ bigger than the length of all linearly independent vectors. 
\item ONLY IF :

For any $m$, we pick m vectors $v_{1}, v_{2}, \cdots, v_{m}$ from $V$, then the $span(v_{1}, v_{2},\cdots,v_{m})\in V$. Then we pickup any $v_{m+1}\in V-span(v_{1}, v_{2},\cdots,v_{m})$, we have vectors $v_{1}, v_{2}, \cdots, v_{m+1}$ are linear independent, based on 2.21.a: if it's linearly dependent, the $v_{m+1}\in span(v_{1}, v_{2},\cdots,v_{m})$.
\end{enumerate}

\subsubsection*{2.A.17}
Copy the answer from \href{https://linearalgebras.com/2a.html}{here}.

Let's think $P_{m}(F)$, the length of its span vectors in $m+1$, so $(z, p_{0}(z), p_{1}(z), \cdots, p_{m}(z))$ is linearly dependent because this list has length $m+2$. 

If $(p_{0}(z), p_{1}(z), \cdots, p_{m}(z))$ is linearly independent, then we have $z\in span(p_{0}(z), p_{1}(z), \cdots, p_{m}(z))$(by problem 11), so we have

\[z+a_{0}p_{0}(z) + a_{1}p_{1}(z) +\cdots + a_{m}p_{m}(z) = 0\]

Since $p_{j}(2) = 0$, we got 2 = 0 in the previously equation, which is not possible.

\subsection*{2.B}
\subsubsection*{2.B.1}

${0}$

\subsubsection*{2.B.3}

\begin{enumerate}[label=(\alph*)]
\item a basis of $U$ is

\[(1, \frac{1}{3}, 0, 0, 0), (0, 0, 1, \frac{1}{7}, 0), (0, 0, 0, 0, 1)\]

\item extend to 

\[(1, \frac{1}{3}, 0, 0, 0), (0, 0, 1, \frac{1}{7}, 0), (0, 0, 0, 0, 1), (0, 1, 0, 0, 0), (0, 0, 0, 1, 0)\]

\item 

\[W = {(0, x, 0, y, 0) \in R^{5} x, y \in R}\]

\end{enumerate}

\subsubsection*{2.B.4}

\begin{enumerate}[label=(\alph*)]
\item a basis of $U$ is

\[(1, 6, 0, 0, 0), (0, 0, 1, \frac{1}{2}, 0), (0, 0, 1, 0, \frac{1}{3})\]

\item extend to 

\[(1, 6, 0, 0, 0), (0, 0, 1, \frac{1}{2}, 0), (0, 0, 1, 0, \frac{1}{3}), (0, 1, 0, 0, 0), (0, 0, 1, 0, 0)\]

\item 

\[W = {(0, x, y, 0, 0) \in R^{5} x, y \in R}\]

\end{enumerate}

\subsubsection*{2.B.5}

The basis does not exists because we have the length of independent list is $\leq$ length of spanning list, here we have a spanning list

\[(1, 0, 0, 0), (0, 0, 1, 0), (0,0,0, 1)\] 

so the list with length 4 can't be a independent list.

\subsubsection*{2.B.6}

\[v_{1}+v_{2}, v_{2}+v_{3}, v_{3}+v_{4}, v_{4}\]

Independent, see $2.A.6$

Spanning list

\[ \forall v\in V, v =  a_{1}v_{1} + a_{2}v_{2} + a_{3}v_{3} + a_{4}v_{4}\]
we also have 
\[ v = a_{1}(v_{1} + v_{2}) + (a_{2} - a_{1})(v_{2} + v_{3}) + (a_{3} -a_{2}+a_{1})(v_{3}+v_{4}) + (a_{4} - a_{3} + a_{2}-a_{1})v_{4} \]

so the list spans $V$.

\subsubsection*{2.B.7}

counterexample

consider this vector list:

\[(1,0,0,0), (0,1,0,0),(0,0,1,0),(0,0,0,1)\]

apparently, the list spans $F^{4}$. Consider $U=span((1,0,0,0), (0,1,0,0), (0,0,1,1)$, we have

$(0,0,1,1) \in U$ but not in $span(v_{1}, v_{2})$.


\subsubsection*{2.B.7}

$u_{1},\cdots, u_{m}, w_{1}, \cdots,w_{n}$ is basis.

\textbf{Independent}

if not independent, then exists
\[a_{1}u_{1}+\cdots+a_{m}u_{m} + b_{1}w_{1} + \cdots +b_{n}w_{n} = 0\]
\[a_{1}u_{1}+\cdots+a_{m}u_{m} = -b_{1}w_{1} - \cdots -b_{n}w_{n}\]

which is conflict the define of direct sum.

\textbf{Spanning list}

$\forall v\in U+W$, $v = u+w\, u\in U, w\in W$

so $\exists a_{1},\cdots,a_{m}, b_{1},\cdots, b_{n}$ that makes

\[ v = a_{1}u_{1}+\cdots+a_{m}u_{m} + b_{1}w_{1} + \cdots +b_{n}w_{n}\]

so $u_{1},\cdots, u_{m}, w_{1}, \cdots,w_{n}$ spans $V$.

\subsection*{2.C}
\subsubsection*{2.C.1}

Let's assume $dim V = n$, so the basis of $U$ is 
\[u_{1}, u_{2}, \cdots, u_{n}\]

If $U\neq V$, then there must be a vector $v\in V-U$, now consider the list

\[u_{1}, u_{2}, \cdots, u_{n}, v\]

it's easy to know that it is independent, but its length is $n+1$ which is larger than $dim V$, so that breaks the law that length of independent $\leq$ length of spanning list which is $n$.

\subsubsection*{2.C.2}

since $R^{2}$ has basis with length 2, so the dim of subspace of $R^{2}$ can be 0, 1 or 2, which is $\{0\}, R^{2}$ or the lines in $R^{2}$. Since $\{0\}$ must be in any subspaces, so all lines must through origin.

\subsubsection*{2.C.4}
\begin{enumerate}[label=(\alph*)]
\item $(x-6), x(x-6), x^{2}(x-6), x^{3}(x-6)$

\item $1, (x-6), x(x-6), x^{2}(x-6), x^{3}(x-6)$
\item $W = span(1)$, so $W=\{c:c \in F\}$
\end{enumerate}

\subsubsection*{2.C.5}
\begin{enumerate}[label=(\alph*)]
\item $1, x, (x-6)^{3}, (x-6)^{4}$
\item $1, x, x^{2}, (x-6)^{3}, (x-6)^{4}$
\item $W = span(x^{2})$, so $W=\{cx^{2}:c \in F\}$
\end{enumerate}

\subsubsection*{2.C.6}
\begin{enumerate}[label=(\alph*)]
\item $1, (x-2)(x-5), x(x-2)(x-5), x^{2}(x-2)(x-5)$
\item $1, x, (x-2)(x-5), x(x-2)(x-5), x^{2}(x-2)(x-5)$
\item $W = span(x)$, so $W=\{cx:c \in F\}$
\end{enumerate}

\subsubsection*{2.C.7}
\begin{enumerate}[label=(\alph*)]
\item $1, (x-2)(x-5)(x-6), x(x-2)(x-5)(x-6)$
\item $1, x, x^{2}, (x-2)(x-5)(x-6), x(x-2)(x-5)(x-6)$
\item $W = span(x, x^{2})$, so $W=\{cx+dx^{2}:c \in F, d \in F\}$
\end{enumerate}

\subsubsection*{2.C.8}
\begin{enumerate}[label=(\alph*)]
\item if $\int^{1}_{-1}f(x)=0$, means $f(x) = -f(-x)$, so the basis of $U$ is $1, x, x^{3}$
\item $1, x, x^{2}, x^{3}, x^{4}$
\item $W = span(x^{4}, x^{2})$, so $W=\{cx^{4}+dx^{2}:c \in F, d \in F\}$
\end{enumerate}

\subsubsection*{2.C.9}

We only have to find a independent list with length $m-1$ in $span(v_{1}+w, \cdots,v_{m}+w)$. The following list is an example:
\[v_{m}-v_{1}, v_{m}-v_{2}, \cdots, v_{m}-v_{m-1}\]

It is independent because $v_{1}, \cdots, v_{m}$ is independent.

\subsubsection*{2.C.10}

We need to prove that the list spans $P(F)$ and it's independent. 

For any $p\in P(F)$, we can start from the scalar of the highest degree $z^{m}$, use $p_{m}$ to guarantee we get the correct $a_{m}$; Then we got the correct $a_{m-1}$ using $z_{m-1}$ in $p_{m}$ and $p_{m-1}$; keep doing this until we got $p_{0}$, so we have the list spans $P(F)$.

In this process, if the target vector is $0$, we can get that all the scalars are $0$, so the list is independent.

\subsubsection*{2.C.11}

we have 
\[dim(U+W) = dim(U) + dim(W) - dim(U\cap W)\]
so we have $dim(U\cap W)=0$, which means the two subspaces has no common vectors except $0$, then we have $U\oplus W$.

\subsubsection*{2.C.12}
\[dim(U\cap W) = 5 + 5 - 9 = 1\]
so $U\cap W \neq \{0\}$

\subsubsection*{2.C.13}

\[dim(U\cap W) = 4 + 4 -6 = 2\]
so if $v_{1}, v_{2}$ is a basis of $U\cap W$, then they are independent, so neither of $v_{1}, v_{2}$ is a scalar multiple of the other.

\subsubsection*{2.C.14}

Make $U_{1\_m} = U_{1} + \cdots + U_{m}$, then we have 
\[dim(U_{1\_m}) \leq dim(U_{1\_m-1})+dim(U_{m}) \leq dim(U_{1\_m-2}) + dim(U_{m-1}) + dim(U_{m}) \leq \cdots \leq dim U_{1} + \cdots + dim U_{m}\]

\subsubsection*{2.C.15}

If $v_{1}, \cdots , v_{n}$ is a basis of $V$, then $\forall 1\leq i, j \leq n$, we have $v_{i}, v_{j}$ are independent. So $span(v_{i}) \cap span(v_{j})=\{0\}$ and $v_{i}, v_{j}$ is the basis of the subspace.

If we add $U_{i}=span(v_{i}$ together, we got a $n$ dimension vector space, with $v_{1}, \cdots, v_{n}$ as its basis, which is $V$.

\subsubsection*{2.C.16}

Make $U_{1\_m} = U_{1} \oplus \cdots \oplus U_{m}$, then we have 
\[dim(U_{1\_m}) = dim(U_{1\_m-1})+ dim(U_{m}) = dim(U_{1\_m-2}) + dim(U_{m-1}) + dim(U_{m}) = \cdots = dim U_{1} + \cdots + dim U_{m}\]
\newpage